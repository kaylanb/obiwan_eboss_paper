% mnras_template.tex
%
% LaTeX template for creating an MNRAS paper
%
% v3.0 released 14 May 2015
% (version numbers match those of mnras.cls)
%
% Copyright (C) Royal Astronomical Society 2015
% Authors:
% Keith T. Smith (Royal Astronomical Society)

% Change log
%
% v3.0 May 2015
%    Renamed to match the new package name
%    Version number matches mnras.cls
%    A few minor tweaks to wording
% v1.0 September 2013
%    Beta testing only - never publicly released
%    First version: a simple (ish) template for creating an MNRAS paper

%%%%%%%%%%%%%%%%%%%%%%%%%%%%%%%%%%%%%%%%%%%%%%%%%%
% Basic setup. Most papers should leave these options alone.
\documentclass[a4paper,fleqn,usenatbib]{mnras}

% MNRAS is set in Times font. If you don't have this installed (most LaTeX
% installations will be fine) or prefer the old Computer Modern fonts, comment
% out the following line
\usepackage{newtxtext,newtxmath}
% Depending on your LaTeX fonts installation, you might get better results with one of these:
%\usepackage{mathptmx}
%\usepackage{txfonts}

% Use vector fonts, so it zooms properly in on-screen viewing software
% Don't change these lines unless you know what you are doing
\usepackage[T1]{fontenc}
\usepackage{ae,aecompl}


%%%%% AUTHORS - PLACE YOUR OWN PACKAGES HERE %%%%%

% Only include extra packages if you really need them. Common packages are:
\usepackage{graphicx}	% Including figure files
\usepackage{amsmath}	% Advanced maths commands
\usepackage{amssymb}	% Extra maths symbols
\usepackage{multicol}        % Multi-column entries in tables
\usepackage{bm}		% Bold maths symbols, including upright Greek
\usepackage{pdflscape}	% Landscape pages
\usepackage{natbib}
%\usepackage{usenatbib}

%%mine
%\usepackage{float}
\usepackage[caption = false]{subfig}  %multiple pngs for one figure
%\usepackage[demo]{graphicx}

\usepackage{trace}
\usepackage{listings}  %upgrade of "verbatim", allows text wrapping
\usepackage{alltt} %non-compiled text tool
\usepackage[normalem]{ulem} %allows underlined text wrap around to next line if too long
%\usepackage{hyperref} % bibtex fields hyperlinks
\usepackage{tablefootnote} %makes \footnote{} work when inside table (places at bottom page)
\usepackage{threeparttable} %need this so where put in table notes actually works

\usepackage[dvipsnames]{xcolor}
\newcommand{\red}[1]{{\textcolor{red}{[#1]}}}
\newcommand{\blue}[1]{{\textcolor{blue}{[#1]}}}
%\newcommand{\red}[1]{{\textcolor{red}{[#1]}}}
\newcommand{\magenta}[1]{{\textcolor{Magenta}{[#1]}}}
\newcommand{\aqua}[1]{{\textcolor{Aquamarine}{[#1]}}}
\newcommand{\green}[1]{{\textcolor{LimeGreen}{[#1]}}}

%%%%%%%%%%%%%%%%%%%%%%%%%%%%%%%%%%%%%%%%%%%%%%%%%%

%%%%% AUTHORS - PLACE YOUR OWN COMMANDS HERE %%%%%

\newcommand{\p}{\partial}
\newcommand{\lp}{\left(}
\newcommand{\rp}{\right)}
\newcommand{\lb}{\left[}
\newcommand{\rb}{\right]}
\newcommand{\Lfig}{(\textit{Left})}
\newcommand{\Rfig}{(\textit{Right})}
\newcommand{\Tfig}{(\textit{Top})}
\newcommand{\Bfig}{(\textit{Bottom})}

\newcommand{\Nsrc}{N_{\rm{src}}}
\newcommand{\Nsky}{N_{\rm{sky}}}
\newcommand{\Nout}{N_{\rm{out}}}
\newcommand{\Rout}{R_{\rm{out}}}
\newcommand{\Ndith}{N_{\rm{dith}}}
\newcommand{\Nexp}{N_{\rm{exp}}}
\newcommand{\Nread}{N_{\rm{read}}}
\newcommand{\Fsrc}{F_{\rm{src}}}
\newcommand{\Fsky}{F_{\rm{sky}}}
\newcommand{\msky}{m_{\rm{sky}}}
\newcommand{\msrc}{m_{\rm{src}}}
\newcommand{\Atele}{A_{\rm{tele}}}
\newcommand{\Apix}{A_{\rm{pix}}}
\newcommand{\Neff}{N_{\rm{eff}}}
\newcommand{\Npix}{N_{\rm{pix}}}
\newcommand{\Ps}{P_{\rm{sc}}}
\newcommand{\texp}{t_{\rm{exp}}}
\newcommand{\tmin}{t_{\rm{min}}}
\newcommand{\tmax}{t_{\rm{max}}}
\newcommand{\texpmin}{t_{\rm{exp, min}}}
\newcommand{\texpmax}{t_{\rm{exp, max}}}
\newcommand{\seeing}{\sigma_{\rm{see}}}
\newcommand{\rhalf}{r_{\rm{half}}}
\newcommand{\magsrc}{m_{\rm{src}}}
\newcommand{\zpi}{ZP_i}
\newcommand{\ebv}{E(B-V)}
\newcommand{\toverhead}{t_{\rm{overhead}}}
\newcommand{\tread}{t_{\rm{read}}}
\newcommand{\tslew}{t_{\rm{slew}}}
\newcommand{\thexapod}{t_{\rm{hexapod}}}
\newcommand{\tdeg}{t_{\rm{deg}}}
\newcommand{\tdone}{t_{\rm{done}}}
\newcommand{\tperfect}{t_{\rm{perfect}}}
\newcommand{\tneed}{t_{\rm{need}}}
\newcommand{\tneedi}{t_{\rm{need, i}}}
\newcommand{\texpi}{t_{\rm{exp, i}}}
\newcommand{\tneedj}{t_{\rm{need, j}}}
\newcommand{\Nslow}{N_{\rm{eff \, < \,3}}}
\newcommand{\Nall}{N_{\rm{all}}}
\newcommand{\Npts}{N_{\rm{pts}}}

\newcommand{\texpo}{t_{\rm{exp, 0}}}
\newcommand{\Fsrco}{F_{\rm{src, 0}}}
\newcommand{\Fskyo}{F_{\rm{sky, 0}}}
\newcommand{\Apixo}{A_{\rm{pix, 0}}}
\newcommand{\Neffo}{N_{\rm{eff, 0}}}
\newcommand{\Kco}{K_{\rm{co}}}
\newcommand{\Aco}{A_{\rm{co}}}
\newcommand{\zpo}{ZP_0}
\newcommand{\zpt}{ZP}
\newcommand{\mskyo}{m_{\rm{sky, 0}}}
\newcommand{\msrco}{m_{\rm{src, 0}}}
\newcommand{\mdesione}{m_{\rm{desi, 1}}}
\newcommand{\mdesitwo}{m_{\rm{desi, 2}}}

\newcommand{\gdecam}{g_{\rm{decam}}}
\newcommand{\rdecam}{r_{\rm{decam}}}
\newcommand{\zdecam}{z_{\rm{decam}}}
\newcommand{\zmosaic}{z_{\rm{mosaic}}}
\newcommand{\gps}{g_{\rm{ps1}}}
\newcommand{\rps}{r_{\rm{ps1}}}
\newcommand{\zps}{z_{\rm{ps1}}}
\newcommand{\gi}{\gps - \rps}
\newcommand{\maperture}{m_{\rm{ap}}}

\newcommand{\ith}{i^{\rm{th}}}
\newcommand{\jth}{j^{\rm{th}}}

\newcommand{\seff}{\text{Survey}_{\rm{ineff}}}
\newcommand{\totalobs}{T_{\rm{obs}}}
\newcommand{\totalreobs}{T_{\rm{reobs}}}
\newcommand{\totalneed}{T_{\rm{need}}}

\newcommand{\logten}{\log_{\rm{10}}}
\newcommand{\Ne}{N_{\rm{e-}}}
\newcommand{\Nskye}{N_{\rm{sky, \, e-}}}
\newcommand{\imageskysub}{\text{Image} - \text{Sky}_{\rm{interp}}}
\newcommand{\mAB}{m_{\rm{AB}}}
\newcommand{\mskyAB}{m_{\rm{sky, AB}}}
\newcommand{\PSmag}{m_{\rm{PS1}}}
\newcommand{\RAgaia}{RA_{\rm{gaia}}}
\newcommand{\DECgaia}{DEC_{\rm{gaia}}}
\newcommand{\RAgood}{RA_{\rm{good}}}
\newcommand{\DECgood}{DEC_{\rm{good}}}
\newcommand{\radiff}{\Delta \text{Ra}}
\newcommand{\decdiff}{\Delta \text{Dec}}
\newcommand{\decrms}{\sigma_{\Delta \text{Dec}}}
\newcommand{\rarms}{\sigma_{\Delta \text{Ra}}}

\newcommand{\skycounts}{\text{Sky}_{e-}}
\newcommand{\skymag}{m_{\rm{sky}}}
\newcommand{\sigmasky}{\sigma_{\rm{sky}}}
\newcommand{\sigmaskyeff}{\sigma_{\rm{sky, eff}}}
\newcommand{\reltransp}{\text{transp}_{\rm{rel}}}
\newcommand{\phoff}{\text{phoff}}
\newcommand{\arcsectwo}{\text{arcsec}^2}
\newcommand{\Xo}{X_0}
\newcommand{\fwhmCP}{\text{FWHM}_{\rm{CP}}}
\newcommand{\fwhmMoffat}{\text{FWHM}_{\rm{Moffat}}}
\newcommand{\fwhmo}{\text{FWHM}_0}
\newcommand{\mdepth}{m_{\rm{depth}}}

\newcommand{\gb}{$g$}
\newcommand{\rband}{$r$}
\newcommand{\zb}{$z$}

\newcommand{\tractor}{{\tt Tractor}}
\newcommand{\legacypipe}{{\tt Legacypipe}}
\newcommand{\obiwan}{{\tt Obiwan}}
\newcommand{\sextractor}{{\tt Source Extractor}~}
\newcommand{\psfex}{{\tt PSFex}}
\newcommand{\sersic}{Sersic}
\newcommand{\dev}{de Vaucouleurs}


% Please keep new commands to a minimum, and use \newcommand not \def to avoid
% overwriting existing commands. Example:
%\newcommand{\pcm}{\,cm$^{-2}$}	% per cm-squared

%%%%%%%%%%%%%%%%%%%%%%%%%%%%%%%%%%%%%%%%%%%%%%%%%%

%%%%%%%%%%%%%%%%%%% TITLE PAGE %%%%%%%%%%%%%%%%%%%

% Title of the paper, and the short title which is used in the headers.
% Keep the title short and informative.
\title[Image Simulations for eBOSS]{Removing Imaging Systematics from eBOSS using {\tt{Obiwan}}}

% The list of authors, and the short list which is used in the headers.
% If you need two or more lines of authors, add an extra line using \newauthor

\author[K. J. Burleigh]{
Kaylan J. Burleigh,$^{1,2}$ \thanks{E-mail: kaylanb@berkeley.edu (KJB)}
Johan Comparat,$^{3,4}$
Hui Kong,$^{5}$
John Moustakas,$^{6}$ 
\newauthor
Anand Raichoor,$^{7,8}$
Ashley Ross$^{5}$
\& David Schlegel$^{2}$
%Peter E. Nugent$^{1,2}$
%\newauthor
%Anna Patej, John Moustakas,$^{5}$, David J. Schlegel$^{2}$, Eddie F. Schlafly$^{2}$, 
%\newauthor and the Legacy Survey Teams
\\
%% List of institutions
$^{1}$Department of Astronomy, University of California at Berkeley, 501 Campbell Hall \#3411, Berkeley, CA 94720, USA\\
$^{2}$Lawrence Berkeley National Laboratory, One Cyclotron Road, Berkeley, CA 94720, USA\\
$^{3}$Departamento de Fisica Teorica, Universidad Autonoma de Madrid, Cantoblanco E-28049, Madrid, Spain \\
$^{4}$Instituto de F\'{i}sica Te\'{o}rica, (UAM/CSIC), Universidad Aut\'{o}noma de Madrid, Cantoblanco, E-28049 Madrid, Spain \\
$^{5}$Department of Physics, Ohio State University, 191 West Woodruff Avenue, Columbus, Ohio 43210, USA \\ 
$^{6}$Department of Physics \& Astronomy, Siena College, 515 Loudon Road, Loudonville, NY, USA 12211 \\
$^{7}$Institute of Physics, Laboratory of Astrophysics, Ecole Polytechnique F\'{e}d\'{e}rale de Lausanne (EPFL), Observatoire de Sauverny, CH-1290 Versoix, Switzerland \\
$^{8}$CEA, Centre de Saclay, IRFU/SPP, F-91191 Gif-sur-Yvette, France \\
}
%$^{4}$Department of Astronomy \& Astrophysics and Dunlap Institute, University of Toronto, Toronto, ON, M5S 3H4, Canada \\
%$^{5}$Department of Physics and Astronomy, Siena College, Loudonville, NY 12211, USA
%}


% These dates will be filled out by the publisher
\date{Accepted YYY. Received YYYY; in original form YYYY}

% Enter the current year, for the copyright statements etc.
\pubyear{2018}

% Don't change these lines
\begin{document}
\label{firstpage}
\pagerange{\pageref{firstpage}--\pageref{lastpage}}
\maketitle

% Abstract of the paper
\begin{abstract}
\begin{itemize}
\item We present the first application of \obiwan, a new method for removing imaging systematics from the DECaLS data (Burleigh \& Moustakas 2018, in prep), to remove these systematics from angular correlation functions.
\item Our goal is to measure the angular correlation function for eBOSS ELGs. This is a crucial test of how well \obiwan\, can improve BAO measurements. The DESI analysis will be more complicated than eBOSS, so this is a necessary step towards a proper scientific analysis for DESI.
\item We perform a trial run on X deg$^2$ of the DR5 footprint, and then run \obiwan\, on the actual eBOSS footprint.
\item We estimate the impact of \obiwan\, on the angular correlation function
\end{itemize}
\end{abstract}

% Select between one and six entries from the list of approved keywords.
% Don't make up new ones.
\begin{keywords}
methods: observational
\end{keywords}

%%%%%%%%%%%%%%%%%%%%%%%%%%%%%%%%%%%%%%%%%%%%%%%%%%

%%%%%%%%%%%%%%%%% BODY OF PAPER %%%%%%%%%%%%%%%%%%

\section{Introduction}

The imaging imaging systematics in SDSS turned out to be fairly independent of one another, so they could be removed with linear models for them (ashley's papers). However, the imaging systematics in the next generation of galaxy surveys (astronomy data sets) will be highly correlated. The data will come from a joint analysis of multiple telescopes' data and/or will have data of the same part of the sky collected over long timelines (months or years).

We want the measure the Baryonic Acoustic Oscillation (BAO) signal, but there are shortcomings in the current methods, such as having apriori knowledge of all the possible imaging systematics in the galaxy survey dataset. DESI will obtain spectra of about 30M galaxies that will be targeted based on multi-band imaging from three telescopes. eBOSS has already targeted and obtained spectra for about \# ELG galaxies, based on DECam data. People have already started confronting this issue, and the previous weight fitting was extended to multiple dimensions (features). Poole et al fit to 25 maps simultaneously.

This paper is the first analysis of large scale data, in this case for eBOSS and DESI, that uses \obiwan\, (Burleigh \& Moustakas 2018, in prep), which is a non-parametric method that corrects for the full covariance of imaging systematics. We perform a trial run on X deg$^2$ of the DR5 footprint, and then run \obiwan\, on the actual eBOSS footprint, restricting to the footprint described in \cite{anand17}. From these runs, we measure the impact of \obiwan\, on the primary science objectives of eBOSS and extrapolate to DESI. All data products are available at NERSC (see Section \ref{sec:data-products}).

\section{Methods}

\subsection{\obiwan}

The five images in Fig. \ref{fig:1000-words} show how Obiwan works. The process is identical to how the Legacy Survey Team produces their Public Data Releases, with one exception. Every time Legacypipe reads an image, we add sim- ulated sources to it. The pipeline doesn't know about the simulated sources so the source detection and model fitting procedure is unaltered. For more details see Burleigh \& Moustakas 2018, in prep. Fig. \ref{fig:examples-faint} show examples of real and simulated ELG-like galaxies with similar \gb \, magnitude that are near the DECaLS detection limit. 
%The ELGs are at our detection limits so visual inspection of single exposure images containing ELGs is challenging.

\begin{figure}
 \includegraphics[width=\columnwidth]{obiwan_1000_words}
 \caption{Example of \obiwan\, adding 4 simulated galaxies to a \gb, \rband, and \zb\, DECam image (200x200 pixels). The original image (top left) is modified by adding 4 simulated galaxies (top middle) to create the new image (top right), which {\tt{Tractor}} operates on. The model and residual images are the bottom panels. Two of the simulated galaxies (the more compact ones) have \dev profiles, while the other two have exponential profiles.}
 \label{fig:1000-words}
\end{figure}

\begin{figure}
 \includegraphics[width=\columnwidth]{fake_real_mosaic_istart_254}
 \caption{Real and simulated ELG-like galaxies near the DECaLS \gb \, detection limit, with similar \gb\, magnitude. The label for each image is on the left (R for real and F for fake or simulated) and its corresponding g-band magnitude is the number on the right. Each row is a single galaxy. The first column is a three color jpeg for easy visualization. The remaining columns are the per-band full resolution coadds for the \gb, \rband, \zb\, images and associated inverse variance maps. Consecutive rows of R and F (rows 1 and 2, 3 and 4, etc.) have similar g-band magnitude for a fair comparison. REPLACE WITH GALAXIES AT eBOSS G=22.9 LIMIT}
 \label{fig:examples-faint}
\end{figure} 

\subsection{Randoms}

We add fake galaxies to the images at random locations, run \legacypipe, and repeat until there are many more simulated galaxies than real ones. The simulated galaxies that \legacypipe\, detects and models are the ``randoms" that a correlation function estimator would use. These randoms have been modified by the full covariance between all imaging systematics, those we expect such as stellar density and seeing, those previously unknown, and the biases and systematics inherit to the \legacypipe\, pipeline itself. 

Our simulated galaxies must be a representative sample of real ELGs or else we cannot use them as randoms. The next section describes how we generate a representative sample by comparing \tractor\, catalogues from DR5 to  spectroscopically confirmed ELG galaxies from DEEP2.  

\subsection{DEEP2}

In this paper, we only concern ourselves with Emission Line Galaxies (ELGs). These are blue star forming galaxies that are spectroscopically identified by a strong OII emission line. The DEEP2 Galaxy Redshift Survey (DEEP2, references) collected one of the largest samples of ELG spectra to date, since the spectral resolution resolves the OII emission line. To model the real population of ELGs, we match our DR3 Tractor catalogues to the DEEP2 catalogues using a 1\arcsec matching radius.

This gives us a sample of ELGs with high confidence redshift (from DEEP2 spectra), brightness (g, r, z flux), and shape (half light radii, ellipticity) information. While the sample is biased by the DEEP2 selection function, the depth of DR5 imaging, and \tractor�s ability to measure the centroid, brightness, and shape, this is as good as we can do with publicly available data. The best possible sample is probably to use Hyper Suprime-Cam Subaru Survey (HSC-SS) imaging in the DEEP2 fields (Hayashi et al. 2017, Mehta et al. 2017).

We model the redshift-brightness-shape parameter space with Gaussian Mixture Models (GMMs). DESI and eBOSS have different ELG target selection criteria so we create separate DESI and eBOSS samples. We then divide each of those into a \dev and Exponential galaxy sample, because we find that \dev, in all bands, are larger and one mag brighter than Exponential, on average. 

\section{X deg$^2$ of DR5}

We inject simulated galaxies into X deg$^2$ of DR5 imaging at density of X galaxies / deg$^2$. About X\% of the simulated galaxies are DESI ELGs, so this is X times larger than DESI target density. 

\subsection{Simulated Galaxy Sample}

We match DR3 to DEEP2 cutting to only spectroscopically confirmed galaxies. 

\dev comprise less than 3\% of the DESI sample, and do not model them because the sample size is too small to capture the underlying distributions in redshift and brightness. For DESI, we apply the following cuts,

{\it{ALL}}
\begin{itemize}
\item flux $>$ 0, ivar $>$ 0
\item rename ``EXP, REX, and SIMP'' as EXP
\item rename ``PSF" as EXP but with rhalf= avg(psfsize g,r,z) / 2
\item rhalf $<$ 5
\end{itemize}

{\it{DESI + EXP (DEV, COMP dropped)}}
\begin{itemize}
\item redshift >= 0.8-0.2 and redshift <= 1.4 + 0.2, so we don�t fit a gaussian to a step function
\item 0.5 mag padding perpendicular to the TS box
\item 0.5 mag deeper than than g,r,z depth limits
\item fit GMM to [fwhm-OR-rhalf, g-r, r-z,z,redshift], 8 components doesn't overfit but captures correlations between variables
\item draw N points, drop points with LogicalOr (0.8 < redshift > 1.4, more than 0.5 deeper than g, r, z limits, and rhalf > pixscale/2), redraw as many points as dropped, repeat until have N points
\end{itemize}

\subsection{Brightness and shape with the appropriate n(z)}

We model the DESI n(z) for ELGs by over-fitting a GMM model to the desired n(z) for DESI. We randomly sampled 100,000 points from the Sanchez \& Kirkby ELG mocks, then fit a 12 component GMM to the resulting n(z) since this reproduced the n(z) very well.

The procedure for generating the sample of ELGs with representative properties for DESI is as follows. We have GMM models for n(z) and the joint distribution of redshift, brightness, and shape. Draw N samples from the n(z) model. Draw 10k samples from the redshift, brigthness, shape model, filtering out points outside the redshift or detection limits. Organize the 10k samples in a KD Tree. For each of the N samples, find the nearest redshift in the 10k sample, and assign the corresponding brightness and shape. This becomes the sample we inject into the imaging data. It has the exact n(z) for DESI or eBOSS and reasonable brightnesses and shapes for each redshift. 42\% of the DESI sample is in the DESI TS box. Note, we repeated the above procedure using DR2 and DR3, respectively, instead of DR5 and ended up with similar Gaussian Mixture Models.

\section{eBOSS NGC and SGC}

We inject simulated galaxies into all DECaLS imaging (DR3 and ``DR3-plus") used to select ELG targets by \cite{anand17}. We inject the galaxies at a density of X galaxies / deg$^2$. About X\% of the simulated galaxies are eBOSS ELGs, so this is X times larger than eBOSS target density. 

\begin{figure}
     \subfloat[NGC\label{fig:ccds-ngc}]{%
       \includegraphics[width=\columnwidth]{ccdsused_eboss_ngc}
     }
     \hfill
     \subfloat[SGC\label{fig:ccds-sgc}]{%
       \includegraphics[width=\columnwidth]{ccdsused_eboss_sgc}
     }
\caption{The eBOSS NGC and SGC footprints (blue boxes) and the DECaLS CCDs from the \tractor\, catalogues that used to select ELG targets.}
\label{fig:ccds-eboss}
\end{figure}

\subsection{ELGs Selected By eBOSS}
\label{sec:eboss-targets}

We use the joint tables (provided by Anand) of \tractor \, catalogue measurements for each eBOSS ELG target and the eBOSS spectra obtained for that target,
\begin{itemize}
\item \verb|eBOSS.ELG.obiwan.eboss21.v5_10_4.fits|
\item \verb|eBOSS.ELG.obiwan.eboss2122.v5_10_7.fits| 
\item \verb|eBOSS.ELG.obiwan.eboss22.v5_10_4.fits|
\item \verb|eBOSS.ELG.obiwan.eboss23.v5_10_4.fits|
\item \verb|eBOSS.ELG.obiwan.eboss23.v5_10_7.fits|
\end{itemize}

\noindent to build a Gaussian Mixture Model for the brightness, shape, and redshift of eBOSS ELG targets. We assume that all galaxies that \tractor \, classifies as type PSF are compact or unresolved. We reclassify them as EXP with $\rhalf = \text{avg}(\text{FWHM}) / 2)$, where the average is over all bands. We also reclassify SIMP sources as EXP. We drop COMP sources because these comprise less than 1\% of the sample. 
%keep galaxies having reasonable \tractor\, measured $\rhalf$.

DEV galaxies are systematically larger and about 1 mag brighter than EXP in all bands (see Fig. \ref{fig:dev-brighter}), so we split the above sample into separate DEV and EXP samples. Fig. \ref{fig:ngc-incomplete} also shows that the NGC region is not complete for the \gb \,mag of the eBOSS ELGs, so we throw out all the NGC galaxies (about 30\% of the sample). This yields 77,525 EXP and 7,439 DEV galaxies. We refer to this as our eBOSS sample. The full list of cuts we apply is:
%9\% of the original sample is DEV, but they fill in parameter space enough for use to model. 
\begin{itemize}
\item !NGC
\item \verb|z_flag == 1| 
\item $0 \le \text{redshift} \le 2$
\item \verb|brick_primary|
\item $\text{type} \neq \text{COMP}$
\item $\rhalf  > 0.131$\arcsec\, (Nyquist limit, one half of the DECam pixel scale)
\item (EXP) $\rhalf  < 2.5$\arcsec
\item (DEV) $\rhalf  < 5.0$\arcsec
\end{itemize}

\begin{figure}
 \includegraphics[width=\columnwidth]{dev_brighter}
 \caption{PDFs of \gb, \rband, \zb mag (top) and redshift and $\rhalf$ (bottom) for eBOSS ELG targets that are spectroscopically confirmed to be galaxies. DEV galaxies (red) are systematically larger and about 1 mag brighter than EXP galaxies, in all bands. We reclassify PSF sources as EXP with $\rhalf = \text{avg}(\text{FWHM}) / 2)$, where the average is over all bands.}
 \label{fig:dev-brighter}
\end{figure}

\begin{figure}
 \includegraphics[width=5 cm]{ngc_incomplete}
 \caption{The eBOSS NGC footprint is not deep enough to detect eBOSS ELGs with $g < 22.825$ mag. We remove galaxies in the NGC region from our sample of representative ELGs.}
 \label{fig:ngc-incomplete}
\end{figure}

Because the SGC \gb \, mag PDF in Fig \ref{fig:ngc-incomplete} is a step function for $g > 22.825$, a Gaussian Mixture model for the brightness, shape, and redshift distribution of eBOSS ELGs will be inaccurate at the faint \gb \, mag, where most of the ELGs reside. To model the full distribution of ELG properties we bootstrap sample from our eBOSS sample.

\subsection{ELGs Almost Selected By eBOSS}
\label{sec:eboss-almost-targets}

In addition to simulating eBOSS ELGs, we want to test contamination by injecting ELGs just outside of the eBOSS selection boundaries; however, the eBOSS sample does not extend beyond the selection boundaries. We create a DR3-DEEP2 matched sample using a 1\arcsec\, matching radius and keeping the nearest match. We reclassify SIMP and PSF sources as EXP the same way we did for the eBOSS sample (see Section \ref{sec:eboss-targets}), and then  apply the following cuts, 
\begin{itemize}
\item !NGC
\item INSERT DEEP2 FLAG THAT SAYS THE OBJECT IS A GALAXY
\item $0 \le \text{redshift} \le 2$
\item \verb|brick_primary|
\item \gb, \rband, \zb $\text{flux} > 0$
\item \gb, \rband, \zb $\text{flux ivar} > 0$
\item $\text{type} \neq \text{COMP}$
\item $\rhalf  > 0.131$\arcsec\, (Nyquist limit, one half of the DECam pixel scale)
\item (EXP) $\rhalf  < 2.5$\arcsec
\item (DEV) $\rhalf  < 5.0$\arcsec
\end{itemize}

We find that the DR3-DEEP2 sample reproduces the brightness, shape, and redshift distributions of the eBOSS sample. Fig. \ref{fig:dr3dp2-reproduces} shows the PDFs of \gb, \rband, \zb, redshift, and $\rhalf$ for EXP galaxies, and that the DR3-DEEP2 galaxies (cut to eBOSS ELG targets) agree very well with the eBOSS sample. Fig. \ref{fig:dr3dp2-redshift} shows that for EXP galaxies, \gb, \rband, \zb, and $\rhalf$ depend on redshift in the same way for the DR3-DEEP2 galaxies (cut to eBOSS ELG targets) and the eBOSS sample. A similar level of agreement occurs for DEV galaxies.

\begin{figure}
 \includegraphics[width=\columnwidth]{dr3dp2_reproduces}
 \caption{PDFs of \gb, \rband, \zb, redshift, and $\rhalf$ for EXP galaxies. The DR3-DEEP2 galaxies (blue), cut to type EXP galaxies, the SGC footprint, and eBOSS ELG targets, reproduces the brightness, shape, and redshift distributions of the eBOSS sample (red).}
 \label{fig:dr3dp2-reproduces}
\end{figure}

\begin{figure}
 \includegraphics[width=\columnwidth]{dr3dp2_redshift_trends}
 \caption{For EXP galaxies, \gb, \rband, \zb, and $\rhalf$ depend on redshift in the same way for the DR3-DEEP2 galaxies (red) and the eBOSS sample (blue). SWAP COLORS SO BLUE IS DR3-DP2 LIKE PREV FIG}
 \label{fig:dr3dp2-redshift}
\end{figure}

Because DR3-DEEP2 galaxies are representative of eBOSS ELGs and extend beyond the eBOSS selection boundaries, we can construct a sample of ELG-like galaxies to test contamination into the eBOSS selected sample. We choose to keep DR3-DEEP2 SGC galaxies within a padding of 0.2 mag of the eBOSSS selection boundaries, yielding a sample of 1,064 EXP and 85 DEV galaxies. Burleigh \& Moustakas 2018, in prep show that \tractor\, flux measurements are accurate to 0.1 - 0.2 mag for \gb $\sim 22.9$ galaxies, so a padding of 0.2 mag should capture most contaminants due to \tractor\, measurement errors. 

\subsection{Simulating eBOSS ELGs with the correct n(z)}

We now describe our algorithm for jointly sampling from the n(z) for eBOSS and from the eBOSS and DR3-DEEP2 samples described in Sections \ref{sec:eboss-targets} and \ref{sec:eboss-almost-targets}. Fig. \ref{fig:dr3dp2-vs-eboss} compares the eBOSS and DR3-DEEP2 samples for EXP and DEV galaxies.

\begin{figure}
     \subfloat[type EXP\label{fig:dr3dp2-vs-eboss-exp}]{%
       \includegraphics[width=\columnwidth]{dr3dp2_vs_eboss_exp}
     }
     \hfill
     \subfloat[type DEV\label{fig:dr3dp2-vs-eboss-dev}]{%
       \includegraphics[width=\columnwidth]{dr3dp2_vs_eboss_dev}
     }
\caption{PDFs of \gb, \rband, \zb mag, $\rhalf$, and redshift for galaxies from the eBOSS (blue) and DR3-DEEP2 (red) samples. (Top) EXP galaxies. (Bottom) DEV galaxies. The DR3-DEEP2 sample includes galaxies within 0.2 mag of the eBOSS ELG target selection boundaries, so it is brighter and fainter than the eBOSS sample. The DR3-DEEP2 DEV are noisy because the sample size is so small (85 galaxies).}
\label{fig:dr3dp2-vs-eboss}
\end{figure}

To build a sample with the correct n(z) we intentionally over-fit a Gaussian Mixture Model to the n(z) from eBOSS spectra. The NGC imaging data are incomplete, so we drop NGC spectra and weight by the spectroscopic completeness (1/TSR in the \verb|eBOSS.ELG.obiwan.eboss*.fits| tables). Fig. \ref{fig:nz} shows the resulting n(z) and that a 10 component GMM reproduces it well. 

\begin{figure}
     \subfloat[eBOSS SGC\label{fig:nz-eboss}]{%
       \includegraphics[width=0.49\columnwidth]{nz_eboss}
     }
     \hfill
     \subfloat[10 component GMM\label{fig:nz-eboss-fit}]{%
       \includegraphics[width=0.46\columnwidth]{nz_eboss_and_thefit}
     }
\caption{(Left) n(z) for eBOSS SGC based on eBOSS 21, 22, and 23 spectra. (Right) Overplotting the PDF from 10,000 draws from our 10 component GMM for n(z), which intentionally over-fits the data.}
\label{fig:nz}
\end{figure}

We now generate a sample of properties for N ELGs that are representative of real ELGs in eBOSS and that \obiwan\, can inject into the DECaLS images. We draw N samples from our n(z) GMM, drop those outside the allowed redshift range [0,2], redraw the number dropped, and repeat until there are N samples. For each of the N redshifts, we find the nearest redshift in the DR3-DEEP2 sample (including EXP and DEV, we using a KD Tree). A) If the DR3-DEEP2 data point is an ELG under the eBOSS ELG target selection, find the nearest redshift in the eBOSS EXP or DEV sample (to the original n(z) redshift). 90\% of the time use the eBOSS EXP sample, and 10\% of the time the eBOSS DEV sample. Assign the associated brightness, shape, and SDSS-ID (\verb|plate-mjd-fiberid|). B) If the DR3-DEEP2 data point is not an ELG under the eBOSS ELG target selection, find the nearest redshift in the DR3-DEEP2 EXP or DEV sample (to the original n(z) redshift). 90\% of the time use the DR3-DEEP2 EXP sample, and 10\% of the time the DR3-DEEP2 DEV sample. Assign the associated brightness, shape, and \tractor-ID (\verb|brickid-objid|). Finally, assign a random RA and Dec coordinate to each of the N ELGs. Fig. \ref{fig:} shows the \gb, \rband, \zb, redshift, and $\rhalf$ distributions for EXP and DEV galaxies resulting from 10,000 draws from the eBOSS n(z) as described above.

\begin{figure}
 \includegraphics[width=\columnwidth]{final_eboss_elg_sample_10k_draws}
 \caption{FIX Representative  for real eBOSS ELGs,  that \obiwan\, will inject into DECaLS images. There are }
 \label{fig:dr3dp2-redshift}
\end{figure}




\subsection{Exploratory Data Analysis}

\begin{figure}
 \includegraphics[width=5 cm]{grz_hist}
 \caption{\gb, \rband, \zb\, magnitude (top to bottom) of simulated galaxies with (green) and without (blue) galactic extinction. Extincted sources are fainter and the effect is stronger for bluer bands.}
 \label{fig:grz-hist.png}
\end{figure}

\begin{figure}
 \includegraphics[width=\columnwidth]{e1_e2_input}
 \caption{(Left) Position angle (pa) versus minor to major axis ratio (ba) for simulated galaxies. (Right) The corresponding ellipticity components (e2, e1).}
 \label{fig:e1-e2}
\end{figure}

\begin{figure}
 \includegraphics[width=7 cm]{delta_dec_vs_delta_ra}
 \caption{RA and Dec residuals (arcsec) between the injected centroid and \tractor's measurement. There is a 0.4\arcsec offset in both RA and Dec because \obiwan injects sources at the nearest pixel center. We expected this offset to be no larger than the DECaLS pixel scale, 0.262\arcsec / pixel, but (possibly due to co-registry or simultaneous fitting of multiple images) is it apparently 0.4\arcsec\, instead.}
 \label{fig:delta-radec}
\end{figure}

\begin{figure}
 \includegraphics[width=\columnwidth]{number_per_type_input_rec_meas}
 \caption{Barplot of the number of galaxies of each type that were injected, detected by \legacypipe\, (recovered), and classified by \tractor \, (tractor). An equal fraction of exponential and \dev\, galaxies are recovered, but the majority of \dev\, galaxies are misclassified as exponential. Source detection is independent of source type, but \tractor model selection is biased against \dev\, galaxies. Note, type REX are called as EXP in this plot.}
 \label{fig:number-per-type-eboss-ngc}
\end{figure}

\begin{figure}
 \includegraphics[width=\columnwidth]{confusion_matrix_by_type}
 \caption{Confusion matrix showing how the true model for a source (y-axis) gets classified by \tractor\, (x-axis). About 70\% of de Vaucouleurs galaxies are misclassified by Tractor compared to only 5\% for exponential galaxies.}
 \label{fig:confusion-matrix-eboss-ngc}
\end{figure}

\begin{figure}
     \subfloat[NGC\label{fig:number-density-eboss-ngc}]{
       \includegraphics[width=\columnwidth]{heatmap_number_density_eboss_ngc}
     }
     \hfill
     \subfloat[SGC\label{fig:number-density-eboss-sgc}]{
       \includegraphics[width=\columnwidth]{heatmap_number_density_eboss_sgc}
     }
\caption{Number density of simulated galaxies in the NGC and SGC footprints. We inject 2400 galaxies / deg$^2$ in both footprints, of which X\% are eBOSS ELG targets. The eBOSS ELG target densities are 200 and 240 targets / deg$^2$ in the NGC and SGC, respectively, so the density of simulated ELGs is at least X times higher than the target density for real ELGs.}
\label{fig:number-density-eboss}
\end{figure}

\begin{figure}
     \subfloat[NGC\label{fig:frac-recovered-eboss-ngc}]{
       \includegraphics[width=\columnwidth]{heatmap_fraction_recovered_eboss_ngc}
     }
     \hfill
     \subfloat[SGC\label{fig:frac-recovered-eboss-sgc}]{
       \includegraphics[width=\columnwidth]{heatmap_fraction_recovered_eboss_sgc}
     }
\caption{Fraction of injected galaxies recovered by \legacypipe\, (i.e. detected and modeled) in the NGC and SGC footprints. Higher fractions are associated with more CCDs (see Fig. \ref{fig:ccds-eboss}).}
\label{fig:frac-recovered-eboss}
\end{figure}


\section{Two Point Correlation Function}
The two point correlation function, applied to galaxy survey data, says whether the galaxies are clustered or under-dense, relative to a random distribution, at different length scales (i.e. separations between pairs of galaxies). The 2D (angular) correlation function is computed by measuring the positions of galaxies in images of the night sky. The 3D correction function combines the 2D information with the distance to each galaxy from Earth (e.g. measured from a spectrum).

Computing the correlation function reduces to counting all pairs of galaxies with separation falling within a given bin. \cite{landy93} showed that 

\begin{align}
\xi(\Delta s) = \frac{GG(\Delta s) - 2 GR (\Delta s) + RR (\Delta s)}{RR (\Delta s)} \label{eqn:lz93}
\end{align}

\noindent is an unbiased estimator for the correlation function, where galaxy-galaxy (GG), galaxy-random (GR), and random-random (RR) are the pair counts in separation ($\Delta s$) bins. We use a different notation than the literature, GG instead of DD (data-data) and GR instead of DR (data-random), because we want to emphasize that the data are images and we are performing all kinds of analyses to select a potentially biased sample of galaxies. The randoms are a sample of random locations on a unit sphere that are then modified by the same procedure used to select the galaxy sample.   

The resulting correlation function is systematically over- and under-dense at different scales due to correlations between galaxy counts and imaging systematics, such as stellar density, seeing, galactic extinction, etc. (Ross et al papers). The correlation function is the expectation value of the number of galaxy pairs, so it is intuitive to write it as a weighted sum, where each weight represents the fraction of those pairs that are over- or under-represented in the sample. In SDSS, a linear model was fit to each imaging systematic correlation and then used to up- or down-sample the number of galaxies depending on the value of the systematic(s) in the region of the image where the galaxies came from. We refer to this post-processing weighting procedure as the {\it{old method}}. We now describe our {\it{new method}} Monte Carlo simulation method, which does not require this weighting procedure. 

We add fake galaxies to the images at random locations, run {\tt{Tractor}}, and the fake galaxies recovered by the Tractor become the randoms. This is true if and only if the fake galaxies we inject are representative of the real ELG galaxies we are interested in for DESI science. These randoms have been modified by the full covariance between all imaging systematics. In principle, this includes systematics we didn't expect, not just stellar density, seeing, extinction, etc. as identified in previous studies.

To simplify terminology, we use {\it{uniform randoms}} to refer to random positions in the footprint, and {\it{obiwan randoms}} to refer to the subset of uniform randoms that are recovered by {\tt{Tractor}}. We wish to test our method over a large footprint, so we ran \obiwan\, on about 500 deg$^2$ of the DR5\footnote{\url{http://legacysurvey.org/status/}} footprint, and present the resulting angular correlation functions in Section \ref{sec:dr5}. We build on these results in Section \ref{sec:eboss}, presenting correlation functions from running \obiwan on the eBOSS NGC and SGC footprints, excluding eboss25.

\subsection{DR5 Analysis}
\label{sec:dr5}

INSERT HUI's WORK

\subsection{eBOSS NGC and SGC Analysis}
\label{sec:eboss}

INSERT HUI's WORK

\section{Plots to Make}
\begin{itemize}
\item galdepth vs. obi wan measured depth, where does the 20\% from validation fall?
\item n(z) resulting from different depths
\item autocorrelation functions for uniform-uniform rand, obiwan-obiwan rand
\item cross correlate everything: (DD=obiwan-rand, RR=uniform-rand), (DD=DR5, RR=uniform-rand), (DD=DR5, RR=obiwan-rand), compare DR5 correlation with known systematics to obiwan-rand with know systematics 
\item plots: BAO fit to angular CF 
\end{itemize}

\section{Conclusions}

\begin{itemize}
\item this is the way to assess bias/systematics when doing joint analysis of datasets from multiple telescopes
\end{itemize}


\section*{Acknowledgements} 
 
 
%%%%%%%%%%%%%%%%%%%%%%%%%%%%%%%%%%%%%%%%%%%%%%%%%%

%%%%%%%%%%%%%%%%%%%% REFERENCES %%%%%%%%%%%%%%%%%%

% The best way to enter references is to use BibTeX:

\bibliographystyle{mnras}  %{mnras} if file is mnras.bst
\bibliography{bib} % {bib} if file is bib.bib


% Alternatively you could enter them by hand, like this:
% This method is tedious and prone to error if you have lots of references
%\begin{thebibliography}{99}
%\bibitem[\protect\citeauthoryear{Author}{2012}]{Author2012}
%Author A.~N., 2013, Journal of Improbable Astronomy, 1, 1
%\bibitem[\protect\citeauthoryear{Others}{2013}]{Others2013}
%Others S., 2012, Journal of Interesting Stuff, 17, 198
%\end{thebibliography}

%%%%%%%%%%%%%%%%%%%%%%%%%%%%%%%%%%%%%%%%%%%%%%%%%%

%%%%%%%%%%%%%%%%% APPENDICES %%%%%%%%%%%%%%%%%%%%%

\appendix

\section{Data Products}
\label{sec:data-products}

TRANSFER DATA TO eBOSS PROJECT DISK SPACE SAY HOW PEOPLE CAN ACCESS IT

\section{Target Selection and Tractor Catalogue Cuts for ELGs}

We use the following target selection and Tractor catalogue cuts to construct our ELG samples for DESI and eBOSS. The ``baseline'' ELG target selection for DESI is  

For eBOSS SGC it is

The color-color box for NGC target selection is a subset of the above, so we simply use the SGC selection in this paper. For both the DESI and eBOSS sample, we apply the following ``quality assurance'' cuts:

%%%%%%%%%%%%%%%%%%%%%%%%%%%%%%%%%%%%%%%%%%%%%%%%%%


% Don't change these lines
\bsp	% typesetting comment
\label{lastpage}
\end{document}

% End of mnras_template.tex