% mnras_template.tex
%
% LaTeX template for creating an MNRAS paper
%
% v3.0 released 14 May 2015
% (version numbers match those of mnras.cls)
%
% Copyright (C) Royal Astronomical Society 2015
% Authors:
% Keith T. Smith (Royal Astronomical Society)

% Change log
%
% v3.0 May 2015
%    Renamed to match the new package name
%    Version number matches mnras.cls
%    A few minor tweaks to wording
% v1.0 September 2013
%    Beta testing only - never publicly released
%    First version: a simple (ish) template for creating an MNRAS paper

%%%%%%%%%%%%%%%%%%%%%%%%%%%%%%%%%%%%%%%%%%%%%%%%%%
% Basic setup. Most papers should leave these options alone.
\documentclass[a4paper,fleqn,usenatbib]{mnras}

% MNRAS is set in Times font. If you don't have this installed (most LaTeX
% installations will be fine) or prefer the old Computer Modern fonts, comment
% out the following line
\usepackage{newtxtext,newtxmath}
% Depending on your LaTeX fonts installation, you might get better results with one of these:
%\usepackage{mathptmx}
%\usepackage{txfonts}

% Use vector fonts, so it zooms properly in on-screen viewing software
% Don't change these lines unless you know what you are doing
\usepackage[T1]{fontenc}
\usepackage{ae,aecompl}


%%%%% AUTHORS - PLACE YOUR OWN PACKAGES HERE %%%%%

% Only include extra packages if you really need them. Common packages are:
\usepackage{graphicx}	% Including figure files
\usepackage{amsmath}	% Advanced maths commands
\usepackage{amssymb}	% Extra maths symbols
\usepackage{multicol}        % Multi-column entries in tables
\usepackage{bm}		% Bold maths symbols, including upright Greek
\usepackage{pdflscape}	% Landscape pages
\usepackage{natbib}
%\usepackage{usenatbib}

%%mine
%\usepackage{float}
%\usepackage[demo]{graphicx}

\usepackage[caption = false]{subfig}  %multiple pngs for one figure

\usepackage{trace}
\usepackage{listings}  %upgrade of "verbatim", allows text wrapping
\usepackage{alltt} %non-compiled text tool
\usepackage[normalem]{ulem} %allows underlined text wrap around to next line if too long
%\usepackage{hyperref} % bibtex fields hyperlinks
\usepackage{tablefootnote} %makes \footnote{} work when inside table (places at bottom page)
\usepackage{threeparttable} %need this so where put in table notes actually works

\usepackage[dvipsnames]{xcolor}
\newcommand{\red}[1]{{\textcolor{red}{[#1]}}}
\newcommand{\blue}[1]{{\textcolor{blue}{[#1]}}}
%\newcommand{\red}[1]{{\textcolor{red}{[#1]}}}
\newcommand{\magenta}[1]{{\textcolor{Magenta}{[#1]}}}
\newcommand{\aqua}[1]{{\textcolor{Aquamarine}{[#1]}}}
\newcommand{\green}[1]{{\textcolor{LimeGreen}{[#1]}}}

%%%%%%%%%%%%%%%%%%%%%%%%%%%%%%%%%%%%%%%%%%%%%%%%%%

%%%%% AUTHORS - PLACE YOUR OWN COMMANDS HERE %%%%%

\newcommand{\p}{\partial}
\newcommand{\lp}{\left(}
\newcommand{\rp}{\right)}
\newcommand{\lb}{\left[}
\newcommand{\rb}{\right]}
\newcommand{\Lfig}{(\textit{Left})}
\newcommand{\Rfig}{(\textit{Right})}
\newcommand{\Tfig}{(\textit{Top})}
\newcommand{\Bfig}{(\textit{Bottom})}

\newcommand{\Nsrc}{N_{\rm{src}}}
\newcommand{\Nsky}{N_{\rm{sky}}}
\newcommand{\Nout}{N_{\rm{out}}}
\newcommand{\Rout}{R_{\rm{out}}}
\newcommand{\Ndith}{N_{\rm{dith}}}
\newcommand{\Nexp}{N_{\rm{exp}}}
\newcommand{\Nread}{N_{\rm{read}}}
\newcommand{\Fsrc}{F_{\rm{src}}}
\newcommand{\Fsky}{F_{\rm{sky}}}
\newcommand{\msky}{m_{\rm{sky}}}
\newcommand{\msrc}{m_{\rm{src}}}
\newcommand{\Atele}{A_{\rm{tele}}}
\newcommand{\Apix}{A_{\rm{pix}}}
\newcommand{\Neff}{N_{\rm{eff}}}
\newcommand{\Npix}{N_{\rm{pix}}}
\newcommand{\Ps}{P_{\rm{sc}}}
\newcommand{\texp}{t_{\rm{exp}}}
\newcommand{\tmin}{t_{\rm{min}}}
\newcommand{\tmax}{t_{\rm{max}}}
\newcommand{\texpmin}{t_{\rm{exp, min}}}
\newcommand{\texpmax}{t_{\rm{exp, max}}}
\newcommand{\seeing}{\sigma_{\rm{see}}}
\newcommand{\rhalf}{r_{\rm{half}}}
\newcommand{\magsrc}{m_{\rm{src}}}
\newcommand{\zpi}{ZP_i}
\newcommand{\ebv}{E(B-V)}
\newcommand{\toverhead}{t_{\rm{overhead}}}
\newcommand{\tread}{t_{\rm{read}}}
\newcommand{\tslew}{t_{\rm{slew}}}
\newcommand{\thexapod}{t_{\rm{hexapod}}}
\newcommand{\tdeg}{t_{\rm{deg}}}
\newcommand{\tdone}{t_{\rm{done}}}
\newcommand{\tperfect}{t_{\rm{perfect}}}
\newcommand{\tneed}{t_{\rm{need}}}
\newcommand{\tneedi}{t_{\rm{need, i}}}
\newcommand{\texpi}{t_{\rm{exp, i}}}
\newcommand{\tneedj}{t_{\rm{need, j}}}
\newcommand{\Nslow}{N_{\rm{eff \, < \,3}}}
\newcommand{\Nall}{N_{\rm{all}}}
\newcommand{\Npts}{N_{\rm{pts}}}

\newcommand{\texpo}{t_{\rm{exp, 0}}}
\newcommand{\Fsrco}{F_{\rm{src, 0}}}
\newcommand{\Fskyo}{F_{\rm{sky, 0}}}
\newcommand{\Apixo}{A_{\rm{pix, 0}}}
\newcommand{\Neffo}{N_{\rm{eff, 0}}}
\newcommand{\Kco}{K_{\rm{co}}}
\newcommand{\Aco}{A_{\rm{co}}}
\newcommand{\zpo}{ZP_0}
\newcommand{\zpt}{ZP}
\newcommand{\mskyo}{m_{\rm{sky, 0}}}
\newcommand{\msrco}{m_{\rm{src, 0}}}
\newcommand{\mdesione}{m_{\rm{desi, 1}}}
\newcommand{\mdesitwo}{m_{\rm{desi, 2}}}

\newcommand{\gdecam}{g_{\rm{decam}}}
\newcommand{\rdecam}{r_{\rm{decam}}}
\newcommand{\zdecam}{z_{\rm{decam}}}
\newcommand{\zmosaic}{z_{\rm{mosaic}}}
\newcommand{\gps}{g_{\rm{ps1}}}
\newcommand{\rps}{r_{\rm{ps1}}}
\newcommand{\zps}{z_{\rm{ps1}}}
\newcommand{\gi}{\gps - \rps}
\newcommand{\maperture}{m_{\rm{ap}}}

\newcommand{\ith}{i^{\rm{th}}}
\newcommand{\jth}{j^{\rm{th}}}

\newcommand{\seff}{\text{Survey}_{\rm{ineff}}}
\newcommand{\totalobs}{T_{\rm{obs}}}
\newcommand{\totalreobs}{T_{\rm{reobs}}}
\newcommand{\totalneed}{T_{\rm{need}}}

\newcommand{\logten}{\log_{\rm{10}}}
\newcommand{\Ne}{N_{\rm{e-}}}
\newcommand{\Nskye}{N_{\rm{sky, \, e-}}}
\newcommand{\imageskysub}{\text{Image} - \text{Sky}_{\rm{interp}}}
\newcommand{\mAB}{m_{\rm{AB}}}
\newcommand{\mskyAB}{m_{\rm{sky, AB}}}
\newcommand{\PSmag}{m_{\rm{PS1}}}
\newcommand{\RAgaia}{RA_{\rm{gaia}}}
\newcommand{\DECgaia}{DEC_{\rm{gaia}}}
\newcommand{\RAgood}{RA_{\rm{good}}}
\newcommand{\DECgood}{DEC_{\rm{good}}}
\newcommand{\radiff}{\Delta \text{Ra}}
\newcommand{\decdiff}{\Delta \text{Dec}}
\newcommand{\decrms}{\sigma_{\Delta \text{Dec}}}
\newcommand{\rarms}{\sigma_{\Delta \text{Ra}}}

\newcommand{\skycounts}{\text{Sky}_{e-}}
\newcommand{\skymag}{m_{\rm{sky}}}
\newcommand{\sigmasky}{\sigma_{\rm{sky}}}
\newcommand{\sigmaskyeff}{\sigma_{\rm{sky, eff}}}
\newcommand{\reltransp}{\text{transp}_{\rm{rel}}}
\newcommand{\phoff}{\text{phoff}}
\newcommand{\arcsectwo}{\text{arcsec}^2}
\newcommand{\Xo}{X_0}
\newcommand{\fwhmCP}{\text{FWHM}_{\rm{CP}}}
\newcommand{\fwhmMoffat}{\text{FWHM}_{\rm{Moffat}}}
\newcommand{\fwhmo}{\text{FWHM}_0}
\newcommand{\mdepth}{m_{\rm{depth}}}

\newcommand{\gb}{$g$}
\newcommand{\rband}{$r$}
\newcommand{\zb}{$z$}
\newcommand{\grcolor}{$g - r$}
\newcommand{\rzcolor}{$r - z$}

\newcommand{\tractor}{{\tt Tractor}}
\newcommand{\legacypipe}{{\tt Legacypipe}}
\newcommand{\obiwan}{{\tt Obiwan}}
\newcommand{\sextractor}{{\tt Source Extractor}~}
\newcommand{\psfex}{{\tt PSFex}}
\newcommand{\sersic}{Sersic}
\newcommand{\dev}{de Vaucouleurs}
\newcommand{\healpix}{{\tt HEALPIX}}


% Please keep new commands to a minimum, and use \newcommand not \def to avoid
% overwriting existing commands. Example:
%\newcommand{\pcm}{\,cm$^{-2}$}	% per cm-squared

%%%%%%%%%%%%%%%%%%%%%%%%%%%%%%%%%%%%%%%%%%%%%%%%%%

%%%%%%%%%%%%%%%%%%% TITLE PAGE %%%%%%%%%%%%%%%%%%%

% Title of the paper, and the short title which is used in the headers.
% Keep the title short and informative.
\title[Removing Imaging Systematics with \obiwan]{Removing Imaging Systematics from the eBOSS ELG Sample with \obiwan}

% The list of authors, and the short list which is used in the headers.
% If you need two or more lines of authors, add an extra line using \newauthor

\author[K. J. Burleigh]{
Kaylan J. Burleigh,$^{1,2}$ \thanks{E-mail: kaylanb@berkeley.edu (KJB)}
Hui Kong,$^{5}$
Johan Comparat,$^{3,4}$
John Moustakas,$^{6}$ 
\newauthor
Anand Raichoor,$^{7,8}$
Ashley Ross$^{5}$
\& David Schlegel$^{2}$
%Peter E. Nugent$^{1,2}$
%\newauthor
%Anna Patej, John Moustakas,$^{5}$, David J. Schlegel$^{2}$, Eddie F. Schlafly$^{2}$, 
%\newauthor and the Legacy Survey Teams
\\
%% List of institutions
$^{1}$Department of Astronomy, University of California at Berkeley, 501 Campbell Hall \#3411, Berkeley, CA 94720, USA\\
$^{2}$Lawrence Berkeley National Laboratory, One Cyclotron Road, Berkeley, CA 94720, USA\\
$^{3}$Departamento de Fisica Teorica, Universidad Autonoma de Madrid, Cantoblanco E-28049, Madrid, Spain \\
$^{4}$Instituto de F\'{i}sica Te\'{o}rica, (UAM/CSIC), Universidad Aut\'{o}noma de Madrid, Cantoblanco, E-28049 Madrid, Spain \\
$^{5}$Department of Physics, Ohio State University, 191 West Woodruff Avenue, Columbus, Ohio 43210, USA \\ 
$^{6}$Department of Physics \& Astronomy, Siena College, 515 Loudon Road, Loudonville, NY, USA 12211 \\
$^{7}$Institute of Physics, Laboratory of Astrophysics, Ecole Polytechnique F\'{e}d\'{e}rale de Lausanne (EPFL), Observatoire de Sauverny, CH-1290 Versoix, Switzerland \\
$^{8}$CEA, Centre de Saclay, IRFU/SPP, F-91191 Gif-sur-Yvette, France \\
}
%$^{4}$Department of Astronomy \& Astrophysics and Dunlap Institute, University of Toronto, Toronto, ON, M5S 3H4, Canada \\
%$^{5}$Department of Physics and Astronomy, Siena College, Loudonville, NY 12211, USA
%}


% These dates will be filled out by the publisher
\date{Accepted YYY. Received YYYY; in original form YYYY}

% Enter the current year, for the copyright statements etc.
\pubyear{2018}

% Don't change these lines
\begin{document}
\label{firstpage}
\pagerange{\pageref{firstpage}--\pageref{lastpage}}
\maketitle

% Abstract of the paper
\begin{abstract}

\red{ADD TEXT}
\begin{itemize}
\item We present the first application of \obiwan, a new method for removing imaging systematics from the DECaLS data \citep{obiwanMethods} to remove these systematics from angular correlation functions.
\item Our goal is to measure the angular correlation function for eBOSS ELGs. This is a crucial test of how well \obiwan\, can improve BAO measurements. The DESI analysis will be more complicated than eBOSS, so this is a necessary step towards a proper scientific analysis for DESI.
\item We perform a trial run on X deg$^2$ of the DR5 footprint, and then run \obiwan\, on the actual eBOSS footprint.
\item We estimate the impact of \obiwan\, on the angular correlation function
\end{itemize}
\end{abstract}

% Select between one and six entries from the list of approved keywords.
% Don't make up new ones.
\begin{keywords}
methods: observational
\end{keywords}

%%%%%%%%%%%%%%%%%%%%%%%%%%%%%%%%%%%%%%%%%%%%%%%%%%

%%%%%%%%%%%%%%%%% BODY OF PAPER %%%%%%%%%%%%%%%%%%

\section{Introduction}

Astronomers perform galaxy surveys to measure how galaxies cluster at different times in the past. Clustering statistics, such as the three dimensional correlation function projected onto a sphere (the angular correlation function), provide a measure of the expansion rate of the universe and can answer many other fundamental questions about the universe \citep{peebles1980}. Some of the most widely known galaxy surveys include the Automatic Plate Measuring Galaxy Survey (APM) \citep{apmSurvey}, the Sloan Digital Sky Survey (SDSS) \citep{sdssYork}, the WiggleZ Dark Energy Survey (WiggleZ) \citep{wigglezSurvey}, the Baryon Oscillation Spectroscopic Survey (BOSS) \citep{bossSurvey}, and the Extended Baryon Oscillation Spectroscopic Survey (eBOSS) \citep{ebossSurvey}. Images of the night sky are transformed into a large scale structure (LSS) catalogue by passing them through a pipeline that automatically detects and models galaxies and stars in the calibrated images. Galaxies of a specific type and age are selected from the LSS catalogue and the clustering statistics are computed from their positions on the sky. 

Removing biases and systematics due to the imaging data (imaging systematics) is critical for measuring unbiased clustering statistics like the angular correlation function. Map-based methods, such as template subtraction and mode projection \citep{biasInTemplateMethod}, have successfully removed imaging systematics from the SDSS, WiggleZ, BOSS, and eBOSS surveys; however, it is unlikely that these methods, in their current state, will be able to handle the complexities of future and ongoing galaxy surveys, such as the Legacy Surveys. Map-based methods use a pixelization scheme, such as \healpix\, \citep{healpix}, to subdivide the sky into equal-area pixels and then compute various per-pixel quantities: the number of galaxies in the LSS catalogue (data), the average seeing, sky brightness, exposure time, etc. (imaging meta-data) and galactic extinction (foregrounds). The non-data maps are potential imaging systematics and are turned into pixel weight maps (in configuration space) or mode weights (in Fourier space). The weights mimic how the angular selection function samples the true distribution of galaxies yielding the observed LSS catalogue \citep{biasInTemplateMethod}.
% the weights can be chained together by multiplying the weights for all systematics or can be fit to all systematics simultaneously

The two most popular map-based methods are ``template subtraction" \citep{sdss1Systematics, myers06, sdss8Systematics, sdss8Companion, sdss9Systematics, sdss12Systematics, wigglezSelectionFunc, delubacSystematics, qsoDepthExtinction, prakashRegressionTech, myersRegressionTech, elvinpoole} and ``model projection" \citep{rybicki92, tegmark98, uros04, biasInTemplateMethod, leistedt13}. Template subtraction models how the number of galaxies depends on each systematic and subtracts the model from each mode of the data (in Fourier space) or divides by the model (in pixel space). \red{Say some stuff about overfitting}. To prevent overfitting, only the systematics maps with the largest cross correlations with the data are modeled. Mode projection treats the systematic maps as adding noise to each mode in Fourier space or pixels in configuration space, so that values in the data covariance matrix are increased for modes where each systematic map is large. It robustly models the impact of the linear combination of the systematics, but does not include non-linear effects from the systematics. 
% biasInTemplateMethod for why template subtraction method overfits
% leistedt13 is best overview for the mode projection method
% they both do "mode projection" because the pixelized maps are a summation of basis functions (usually spherical harmonics) in Fourier space. The observed datas is the superpostion of cosmological signal modes and systematics modes. Computing a correlation function or power spectrum in Fourier space ``projects" the data onto the basis function's modes. The template subtraction method builds a model for each systematic and subtracts the model in Fourier space (or divides by it in configuration space), while the mode projection method increases the covariance matrix of the data at modes where each systematic is strongest.

These map-based methods are ill suited for the next generation of galaxy surveys, such as the Legacy Surveys \citep{overviewPaper}. A substantial fraction of the data ($10-20$\%) is removed because it contains bright stars, bad seeing, etc. and the number of galaxies deviate significantly from the mean. Only systematics that are known apriori are corrected for and it is unclear how to make a single systematic map for repeat imaging of the same region. For example, \cite{elvinpoole} threw out about 20\% of the imaging data and needed more than 19 systematic maps for DECam imaging data. Any biases or systematics introduced by the pipeline that created the LSS catalogue are ignored. LSS catalogues from the Legacy Surveys require a joint analysis of images from three telescopes. Each telescope will obtain multi- and same-band images of the same part of the sky that are separated by month to year time baselines. The CCD detectors also have similar angular size (X-X deg) to the BAO signal signal (X deg at a redshift of X).
%It is unclear whether map-based methods are capable of removing imaging systematics, especially due to the latter, at a level sufficient for a Stage-IV dark energy science. 
%The complexities of the Legacy Surveys have already revealed shortcomings in the current method for removing imaging systematics. 

We present a new method for removing imaging systematics from future and ongoing surveys that does not require maps of imaging systematics, foregrounds, or other apriori knowledge (i.e. it is non-parametric), and that corrects for biases and systematics in the software pipeline that produced the LSS catalogue. We apply our method to DECam data from the Legacy Surveys using the \obiwan\, code \citep{obiwanMethods}. We inject realistic emission line galaxies (ELGs) into the DECam images used to create for DR3-era \tractor\, catalogues, which the eBOSS Team used to select ELG targets \citep{anand17}. We use \obiwan\, to perform Monte-Carlo simulations of how the \legacypipe/\tractor\, pipeline detects and forward-models eBOSS ELG-like galaxies. Our goal is compute the angular correlation function for eBOSS ELGs with and without \obiwan, to estimate the impact of this method on eBOSS science requirements. This is also a preparatory step towards future analysis of the Dark Energy Spectroscopic Instrument (DESI) ELG sample, because DESI will select targets using Legacy Surveys data and its five-year survey is significantly more complicated than eBOSS \citep{desiScience, desiInstrument}.  All data products are available at NERSC (see Section \ref{sec:data-products}).
%will obtain spectra for about 30M galaxies and all targets will come from the Legacy Surveys 
%We can measure the impact of \obiwan\, on the primary science objectives of surveys like eBOSS and DESI, by computing two-point correlations functions with and without our corrections applied. 

%The model for each systematic predicts how much to up- or down-sample the number of galaxies in a given part of the sky. This method assumes apriori knowledge of the potential imaging systematics and that they are independent of one other. 
%These assumptions will likely break down in the next generation of galaxy surveys, such as the Legacy Surveys (aka DECaLS, MzLS, and BASS) \red{CITE OVERVIEW PAPER}. 
%The Legacy Surveys take images with three telescopes and leverage existing imaging from Wide-field Infrared Survey Explorer (WISE) \red{cite} to detect galaxies. 

This paper is structured as follows. In \S\,\ref{sec:data}, we describe the imaging and spectroscopic data we use and the eBOSS ELG target selection criteria. In \S\,\ref{sec:methods}, we summarize how \obiwan\, and \tractor\, work, the algorithms we use for removing imaging systematics, and the angular correlation function is estimated from a LSS catalogue. In \S\,\ref{sec:results}, we present our \obiwan\, Monte-Carlo simulations of the same imaging data used to select eBOSS ELGs, and the resulting angular correlation functions. We conclude in \S\,\ref{sec:conclusions}. The Appendix provides the additional details about our \obiwan\, Monte-Carlo simulations needed to reproduce our results, but that do not belong in the main text.

\section{Data}
\label{sec:data}

\subsection{The DECam Legacy Survey (DECaLS)}

The DECaLS is a \gb, \rband, \zb-band survey of 9,000 deg$^2$ of the southern sky using the Blanco 4-m telescope and DECam camera\footnote{\url{http://www.ctio.noao.edu/noao/content/DECam-Observing-Manual}} in Cerro Tololo, Chile. DECam has a field of view of 3.18 deg$^2$ and is a mosaic of 62 CCDs, each having 4096x2046 pixels, with pixel scale of 0.262\arcsec pixel$^{-1}$. The DECaLS depth requirements are 1-2 mag deeper than the SDSS. For more details see \cite{overviewPaper, strategyPaper}.

The first round of eBOSS ELG target selection \citep{anand17} used a combination of DR3\footnote{\url{http://legacysurvey.org/dr3}} \tractor\, catalogues and a set of reprocessed DR3 \tractor\, catalogues (produced by the eBOSS team) that included DECam images observed after the DR3 March 2016 cutoff.  We will refer to these as the DR3-plus catalogues. The list of DECam CCDs used to create the DR3-plus catalogues is available online\footnote{\url{http://portal.nersc.gov/project/desi/users/kburleigh/obiwan/legacysurveydir_ebossdr3}}. Fig. \ref{fig:ccds-eboss} shows these CCDs and the approximate eBOSS NGC and SGC regions (blue boxes). 

\begin{figure}
     \subfloat[NGC\label{fig:ccds-ngc}]{%
       \includegraphics[width=\columnwidth]{ccdsused_eboss_ngc}
     }
     \hfill
     \subfloat[SGC\label{fig:ccds-sgc}]{%
       \includegraphics[width=\columnwidth]{ccdsused_eboss_sgc}
     }
\caption{The eBOSS NGC and SGC footprints (blue boxes) and the DECaLS CCDs used to create the DR3-plus \tractor\, catalogues.}
\label{fig:ccds-eboss}
\end{figure}

\subsection{eBOSS ELG Target Selection}

eBOSS selected ELGs from DR3-plus \tractor\, catalogues having clean DECaLS photometry, locations outside bright star masks, sufficient \gb-flux to be [O II] emitters and star forming galaxies, and \grcolor\, and \rzcolor\, color associated with galaxies in the desired redshift range of 0.5 - 2. The eBOSS ELG footprint is split into the two regions (blue boxes) shown in Fig. \ref{fig:ccds-eboss}. The regions include 620 deg$^2$ in the South Galactic Cap (SGC) and 600 deg$^2$ in the North Galactic Cap (NGC). ELGs in the SGC are selected using by the following \tractor\, catalogue cuts,

\begin{itemize}
\item \verb|brick_primary|
\item not in bright star masks
\item \verb|decam_anymask[grz] = 0|
\item $21.825 < g < 22.825$
\item $-0.068\, (r-z) + 0.457 < g-r < 0.112\, (r-z) + 0.773$
\item $0.218\, (g-r) + 0.571 < r-z < -0.555\, (g-r) + 1.901$
\end{itemize}

\noindent The NGC cuts are identical except for,

\begin{itemize}
\item $21.825 < g < 22.9$
\item $0.637\, (g-r) + 0.399 < r-z$
\end{itemize}

\noindent For more details see \cite{anand17}.

It was later discovered that \verb|decam_anymask[grz] = 0| is magnitude dependent \red{also redshift biased?} and is removes many good ELG candidates. \verb|decam_allmask[grz] = 0| should have been used instead. Identifying and removing the biases and systematics introduced \verb|decam_anymask[grz] = 0| has proven difficult; fortunately, our \obiwan\, Monte-Carlo simulations will resolve this problem because they provide a set of randoms with and without the \verb|decam_anymask[grz] = 0| cut applied. 

\subsection{Joint Tables of eBOSS Spectra and \tractor\, Catalogue Measurements}

To build a representative sample of eBOSS ELG galaxies (see \S\, \ref{sec:injecting-elgs}), we use the following joint tables of eBOSS 21, 22, 23 spectra and associated DR3-plus \tractor\, catalogue measurements,

\begin{itemize}
\item \verb|eBOSS.ELG.obiwan.eboss21.v5_10_4.fits|
\item \verb|eBOSS.ELG.obiwan.eboss2122.v5_10_7.fits| 
\item \verb|eBOSS.ELG.obiwan.eboss22.v5_10_4.fits|
\item \verb|eBOSS.ELG.obiwan.eboss23.v5_10_4.fits|
\item \verb|eBOSS.ELG.obiwan.eboss23.v5_10_7.fits|
\end{itemize}

\noindent We will refer to these as the eBOSS-\tractor\, tables. 

\subsection{The DEEP2 Galaxy Redshift Survey (DEEP2)}

DEEP2 obtained about 50,000 high resolution (R $\sim 6000$) spectra of redshift $\sim 1$ galaxies using the DEIMOS multi-object on Keck 2 \citep{deep2}. The DEEP2 footprint is 2.8 deg$^2$, split into four disjoint regions: Field 1 (14hr), Field 2 (16h), Field 3 (23h), and Field 4 (02h). We create a DEEP2 (DR4) and DECaLS DR3 matched table by finding the nearest DR3 \tractor\, catalogue source within a 1\arcsec search radius of each DEEP2 spectrum. The DECaLS DR3 footprint does not overlap Field 1, so our table only includes Fields 2-4. We refer to it as the DR3-DEEP2 table and use it in \S\, \ref{sec:injecting-elgs}.

\section{Methods}
\label{sec:methods}

\subsection{\obiwan}

\obiwan\, modifies the \gb, \rband, \zb\, images that \legacypipe\, operates on by adding simulated sources to the individual exposures and appropriately modifying the inverse variance images. The simulated sources include poisson noise from the source itself. The power of \obiwan\, is that the injected sources inherit the sky background, systematics, or whatever else is present in the data, so nothing more than the simulated galaxy or star of interest is injected. \legacypipe\, does not know the images have been modified and source detection, model fitting, and model selection proceed as usual. For more details see \cite{obiwanMethods}. 

Fig. \ref{fig:1000-words} shows how \obiwan\, works. a 200x200 pixel DECam \gb, \rband, \zb\, coadded image (top left), the coadd of the four simulated galaxies that will be added (top middle), and the image coadd after adding the four galaxies to each exposure (top right). The model and residual coadd images are the bottom panels. Two of the simulated galaxies (the more compact ones) have \dev\, profiles, while the other two have exponential profiles. Fig. \ref{fig:examples-bright} show examples of relatively bright real and simulated ELG-like galaxies with similar \gb \, magnitude.

\begin{figure}
 \includegraphics[width=\columnwidth]{obiwan_1000_words}
 \caption{Example of \obiwan\, adding 4 simulated galaxies to a \gb, \rband, and \zb\, DECam image (200x200 pixels). The original image (top left) is modified by adding 4 simulated galaxies (top middle) to create the new image (top right), which \tractor\, operates on. The model and residual images are the bottom panels. Two of the simulated galaxies (the more compact ones) have \dev profiles, while the other two have exponential profiles.}
 \label{fig:1000-words}
\end{figure}

%\begin{figure}
% \includegraphics[width=\columnwidth]{fake_real_mosaic_istart_254}
% \caption{Real and simulated ELG-like galaxies near the DECaLS \gb \, detection limit, with similar \gb\, magnitude. The label for each image is on the left (R for real and F for fake or simulated) and its corresponding g-band magnitude is the number on the right. Each row is a single galaxy. The first column is a three color jpeg for easy visualization. The remaining columns are the per-band full resolution coadds for the \gb, \rband, \zb\, images and associated inverse variance maps. Consecutive rows of R and F (rows 1 and 2, 3 and 4, etc.) have similar g-band magnitude for a fair comparison. \red{REPLACE WITH GALAXIES AT eBOSS G=22.9 LIMIT}}
% \label{fig:examples-faint}
%\end{figure} 

\begin{figure}
 \includegraphics[width=\columnwidth]{fake_real_mosaic_istart_0}
 \caption{Relatively bright real and simulated exponential galaxies, with similar \gb\, magnitude. The label for each image is on the left (R for real and F for fake or simulated) and its corresponding \gb\, magnitude is the number on the right. Each row is a single galaxy. The first column is a three color jpeg for easy visualization. The remaining columns are the per-band full resolution coadds for the \gb, \rband, \zb\ images and associated inverse variance maps. Consecutive rows of R and F (rows 1 and 2, 3 and 4, etc.) have similar \gb\, magnitude for a fair comparison.}
 \label{fig:examples-bright}
\end{figure} 

\obiwan\, performs a Monte-Carlo Simulation by injecting the simulated galaxies at random RA and Dec, running \legacypipe, and repeating for the same images. In large scale structure terminology, the randoms that would be used in a correlation function analysis are the injected galaxies that \legacypipe\, detects, models, and measures properties for that pass some target selection criteria. The initially random RA and Dec centroids have been modified by the same survey geometry, source detection, measurement, and target selection process that modified the real galaxy targets in the images. Unlike the methods that rely on backward-modeling of apriori imaging systematics, these {\it{obiwan randoms}} have already modified by the full covariance between {\it{all}} imaging systematics, those we expected apriori but also those not expected or unknown. The obiwan randoms are also modified by any biases and systematics in the \legacypipe\, pipeline which are of course in the real galaxy sample as well. 

\subsection{\tractor}

\red{insert few sentences about tractor and its stages.} For more details see \cite{obiwanMethods, tractorPaper}.

\subsection{Injecting Realistic ELGs}
\label{sec:injecting-elgs}

This section summarizes how we generate the representative sample of eBOSS ELG-like galaxies that we inject into the images. A representative sample is crucial to the success of our method because we can only use the randoms if they truly mimic the properties of the real galaxies we are interested in. 

We use the eBOSS-\tractor\, tables to build a sample of eBOSS ELG targets having a redshift, shape, and \gb, \rband, \zb flux. DEV galaxies are systematically larger and about 1 mag brighter than EXP in all bands (see Fig. \ref{fig:dev-brighter}), so we split the sample into separate DEV and EXP samples. We also drop all NGC galaxies (about 30\% of the sample) because the NGC \tractor catalogues are not complete to \gb $< 23.8$ mag (see Fig. \ref{fig:ngc-incomplete}), 0.025 - 0.1 mag brighter than the \gb\, limit for eBOSS ELGs. The final ``eBOSS'' sample has 77,525 EXP and 7,439 DEV galaxies. See Section \ref{a:input-sample} for more details.

The eBOSS sample does not extend beyond the eBOSS ELG selection boundaries, but to simulate contamination we need to inject ELGs just outside the eBOSS ELG selection boundaries. We find that our DR3-DEEP2 matched table reproduces the brightness, shape, and redshift distributions of the eBOSS sample (see Section \ref{a:eboss-almost-targets}). The DR3-DEEP2 galaxies extend beyond the eBOSS ELG selection boundaries, so we construct a sample of ELGs that are within 0.2 mag of the boundaries. The final ``DR3-DEEP2'' sample has 1,064 EXP and 85 DEV galaxies. See Section \ref{a:eboss-almost-targets} for more details.

We now describe our algorithm to jointly sample redshift from the eBOSS n(z) and the associated \gb, \rband, \zb flux and shape from our eBOSS and DR3-DEEP2 samples. The eBOSS n(z) is the redshift distribution from the \verb|eBOSS.ELG.obiwan.eboss*.fits| tables weighted by spectroscopic completeness (1/TSR). We intentionally over-fitting a 10 component Gaussian Mixture Model (GMM) to this n(z) so that we can sample from it. 

We draw redshifts from n(z), dropping those outside the allowed redshift range [0,2], until there are N redshift samples. For each redshift, we find the nearest redshift in our EXP-DEV combined DR3-DEEP2 sample. We define an ELG as passing eBOSS ELG SGC target selection. If the redshift-brightness-shape sample is an ELG, we find its nearest redshift in the EXP eBOSS sample (90\% of the time) or the DEV eBOSS sample (10\% of the time); if not, we trim the DR3-DEEP2 sample to galaxies that extends beyond the eBOSS ELG selection boundaries, and find its nearest redshift in the trimmed EXP DR3-DEEP2 sample (90\% of the time) or the trimmed DEV DR3-DEEP2 sample (10\% of the time). This yields a sample of eBOSS ELG-like galaxies with the desired redshift distributed. See Section \ref{a:final-eboss-sample}.

\subsection{Run \obiwan\, on the DECam CCDs used to Select ELG Targets for eBOSS}

We use \obiwan\, to inject simulated galaxies into the DECam CCDs used to create the DR3-plus \tractor\, catalogues. We use the current version of \legacypipe\, and not the two year-old version that actually created the DR3-era \tractor catalogues. We do not expect this to bias our results because we find excellent agreement between DR3 and DR5 measurements for the same sources. We configure \obiwan\, to run \legacypipe\, with the \verb|--simp| option, which uses the SIMP model instead of REX, and to explicitly use all the input CCDs (used to create the DR3-era \tractor\, catalogues and to select the eBOSS ELG targets).   

\subsection{Using the Angular Correlation Function to Gauge Scientific Impact}

We will compute the angular correlation function, with and without \obiwan\, randoms, to gauge the scientific impact of \obiwan\, on eBOSS science. 

The angular correlation function ($\xi(r)$) is X. Assuming a power law for $\xi(r)$,
\begin{align}
\xi = \lp\frac{r}{r_0}\rp^{-\gamma}, 
\end{align}
where $r_0$ is the characteristic separation between galaxies, and the small angle ($\theta \ll 1 \text{rad} \approx 60 \text{deg}$) limit, the angular correlation function is also a power law \citep{peebles1980}, 
\begin{align}
w = \theta^{1-\gamma}. 
\end{align}
For ELGs, $\gamma \approx 1.6$ and $r_0 \approx 4$ Mpc h$^{-1}$ \red{cite}
%1.7-1.8 for galaxies (conolloy 2002)
%ELG gamma=1.6, r0=4 (Ellie)
%LRG gamma= 1.8, r0=10 (Ellie)
%QSOs gamm= 1.6, r0=6 (Ellie)

The two point correlation function, applied to galaxy survey data, says whether the galaxies are clustered or under-dense, relative to a random distribution, at different length scales (i.e. separations between pairs of galaxies). The 2D (angular) correlation function is computed by measuring the positions of galaxies in images of the night sky. The 3D correction function combines the 2D information with the distance to each galaxy from Earth (e.g. measured from a spectrum).

Computing the correlation function reduces to counting all pairs of galaxies with separation falling within a given bin. \cite{landy93} showed that 

\begin{align}
\xi(\Delta s) = \frac{GG(\Delta s) - 2 GR (\Delta s) + RR (\Delta s)}{RR (\Delta s)} \label{eqn:lz93}
\end{align}

\noindent is an unbiased estimator for the correlation function, where galaxy-galaxy (GG), galaxy-random (GR), and random-random (RR) are the pair counts in separation ($\Delta s$) bins. We use a different notation than the literature, GG instead of DD (data-data) and GR instead of DR (data-random), because we want to emphasize that the data are images and we are performing all kinds of analyses to select a potentially biased sample of galaxies. The randoms are a sample of random locations on a unit sphere that are then modified by the same procedure used to select the galaxy sample.   

The resulting correlation function is systematically over- and under-dense at different scales due to correlations between galaxy counts and imaging systematics, such as stellar density, seeing, galactic extinction, etc. (Ross et al papers). The correlation function is the expectation value of the number of galaxy pairs, so it is intuitive to write it as a weighted sum, where each weight represents the fraction of those pairs that are over- or under-represented in the sample. 

\section{Results}
\label{sec:results}

\subsection{Run \obiwan\, on the DECam CCDs used to Select ELG Targets for eBOSS}

We inject X simulated galaxies (i.e. randoms), at a density of X / deg$^2$ (X / brick) into the DR3-era CCDs. X\% of the injected galaxies are ELGs under the eBOSS ELG target selections, so we are injecting, on average, X ELGs / deg$^2$ which will be reduced further by whichever \legacypipe detects and models. Heatmaps of the number density of simulated galaxies added to each brick are shown in Fig. \ref{fig:number-density-eboss}. 

\begin{figure}
     \subfloat[NGC\label{fig:number-density-eboss-ngc}]{
       \includegraphics[width=\columnwidth]{heatmap_number_density_eboss_ngc}
     }
     \hfill
     \subfloat[SGC\label{fig:number-density-eboss-sgc}]{
       \includegraphics[width=\columnwidth]{heatmap_number_density_eboss_sgc}
     }
\caption{Number density of simulated galaxies in the NGC and SGC footprints. We inject 2400 galaxies / deg$^2$ in both footprints, of which X\% are eBOSS ELG targets. The eBOSS ELG target densities are 200 and 240 targets / deg$^2$ in the NGC and SGC, respectively, so the density of simulated ELGs is at least X times higher than the target density for real ELGs.}
\label{fig:number-density-eboss}
\end{figure}

Fig. \ref{fig:input} shows the \gb, \rband, and \zb\, mag histograms for the simulated galaxies before and after adding galactic extinction (left column), the distribution of injected $\rhalf$ (right top), and the distribution of ellipticity components e1, e2 (right middle). The injected galaxies are 89\% exponential and 11\% \dev\,, and \legacypipe\, recovers 76\% of the exponentials and 73\% of the \dev (see Fig. \ref{fig:number-per-type}). This suggests that \legacypipe\, is equally good at recovering exponential and \dev\, sources.

\begin{figure}
\begin{tabular}{cc}
      \gb, \rband, \zb\, magnitude & $\rhalf$, e1, e2 \\
      % list of figures goes left to right, then down a row, ...
      \includegraphics[width=0.45\columnwidth]{grz_hist_input_ext_g} &
      \includegraphics[width=0.45\columnwidth]{hist_true_rhalf_input}
      \\
      \includegraphics[width=0.45\columnwidth]{grz_hist_input_ext_r} &
      \includegraphics[width=0.45\columnwidth]{e1_e2_input_1panel}
      \\
      \includegraphics[width=0.45\columnwidth]{grz_hist_input_ext_z} &
      \\
\end{tabular}
\caption{(Left) \gb, \rband, \zb\, magnitude of injected sources (green) and after adding galactic extinction to them (blue). Extincted sources are fainter and the effect is stronger for bluer bands. (Right Top) $\rhalf$ of injected sources. The most common size is $\rhalf \sim 0.5$\arcsec. (Right Middle) Ellipticity components, e2 versus e1, of injected sources. These correspond to sampling from uniform distributions of position angles [0, 180) and minor to major axis ratios [0.2, 1.0].  
    \label{fig:input}}
\end{figure}

\begin{figure}
 \includegraphics[width=\columnwidth]{number_per_type_input_rec_meas}
 \caption{Barplot comparing the number of galaxies injected versus recovered by \legacypipe. The injected population is 89\% exponential and 11\% \dev\,, and \legacypipe\, recovers 76\% of the exponentials and 73\% of the \dev. This suggests that \legacypipe\, is equally good at recovering exponential and \dev\, sources.}
 \label{fig:number-per-type}
\end{figure}

The power of \obiwan\, is evident in Fig. \ref{fig:frac-recovered-eboss}. The heat maps show the fraction of injected galaxies that \legacypipe\, recovers. It is no longer necessary to determine N weights for up- and down-sampling using linear fits to galaxy correlations with N systematics. Fig. \ref{fig:frac-recovered-eboss} is the weights, per brick.

\begin{figure}
     \subfloat[NGC\label{fig:frac-recovered-eboss-ngc}]{
       \includegraphics[width=\columnwidth]{heatmap_fraction_recovered_eboss_ngc}
     }
     \hfill
     \subfloat[SGC\label{fig:frac-recovered-eboss-sgc}]{
       \includegraphics[width=\columnwidth]{heatmap_fraction_recovered_eboss_sgc}
     }
\caption{Fraction of injected galaxies that \legacypipe\, recovers. Higher fractions are generally due to having more CCDs (Fig. \ref{fig:ccds-eboss}).}
\label{fig:frac-recovered-eboss}
\end{figure}

\legacypipe\, modifies the simulated galaxy sample in three ways: true positives, false positives, and false negatives. True positives (recovered ELGs) are simulated ELGs that remain eBOSS ELGs using \legacypipe's measurements for them. False positives (contaminants) are simulated non-ELGs that pass target selection after \legacypipe\, measures them. False negatives (lost ELGs) are simulated ELGs that \legacypipe\, fails to detect (not-detected), fail target selection after \legacypipe\, measures them (fail-ts), or are removed by the {\tt{fracin}} cut (fracin-cut; see Section \ref{a:biases-systematics}). The distributions of recovered ELGs, contaminants, and lost ELGs in the eBOSS color box are shown in Fig. \ref{fig:rec-lost-contam-color}. Their \gb, \rband, \zb\, magnitude distributions are shown in Fig. \ref{fig:rec-lost-contam-mag}.

\begin{figure}
 \includegraphics[width=\columnwidth]{rec_lost_contam_gr_rz}
 \caption{Distributions of recovered ELGs, contaminants, and lost ELGs in the eBOSS color box. (Left) true color of source. (Right) \tractor\, measured color. From top to bottom are recovered ELGs (blue), contaminants (green), lost ELGs fail-ts (cyan), lost ELGs not-detected (magenta), and lost ELGs fracin-cut (yellow). Most contaminants come from the upper left region of the color box. The primary reason for loosing ELGs is that the \tractor\, measurements fail target selection.}
 \label{fig:rec-lost-contam-color}
\end{figure}

\begin{figure}
 \includegraphics[width=7 cm]{rec_lost_contam_grz}
 \caption{\gb, \rband, \zb\, magnitude histograms for recovered ELGs, contaminants, and lost ELGs, using the color scheme from Fig. \ref{fig:rec-lost-contam-color}. There are similar number of recovered ELGs and lost ELGs fail-ts for sources brighter than \gb < 22.5, \rband < 21.9, or \zb < 21.0, but there are many more lost ELGs fail-ts for fainter sources than that. The \gb, \rband, \zb\, distributions of contaminants, lost ELGs not-detected, and lost ELGs fracin-in are roughly uniform.}
 \label{fig:rec-lost-contam-mag}
\end{figure}

The ELGs we inject have the appropriate correlations among brightness, shape, and redshift, so we can ask, how does \legacypipe\, modify the input n(z)? The top panel of Fig. \ref{fig:redshifts-recovered} shows that redshifts $< 0.2$ and $> 1.4$ are lost. The bottom panel shows the fraction of contaminants for different redshift bins.

\begin{figure}
 \includegraphics[width=0.7\columnwidth]{redshifts_recovered}
 \caption{(Top) PDF of redshift for injected ELGs (blue) and recovered ELGs (green). (Bottom) The fraction of the injected ELG sample that are recovered ELGs (green) compared to contaminants (magenta).}
 \label{fig:redshifts-recovered}
\end{figure}

\subsection{Using the Angular Correlation Function to Gauge Scientific Impact}

\red{INSERT HUI's WORK}

For computing the angular correlation function: \citep{corrfuncErrors, corrfuncIC}.

\obiwan\, allows us to compute the angular correlation function without using weights to up- or down-sample the randoms. However, \obiwan\, does provide weights per-brick, which existing datasets or future studies can use to up- and down-sample with. For instance, weights could be the fraction of injected ELG galaxies that \legacypipe\, recovers (see Fig. \ref{fig:frac-recovered-eboss}) or the fraction of recovered ELGs that remain after the eBOSS anymask cut. 

\subsection{The eBOSS ``anymask'' cut}

\red{Make a fraction recovered map after applying the anymask cut and show an angular correlation function with and without anymask}

\subsection{Biases and Systematics in \legacypipe}
\label{sec:biases-systematics}

We now present high impact biases and systematics that we find in \legacypipe\, using our \obiwan\, eBOSS runs. See Section \ref{a:biases-systematics} for more detail. Fig. \ref{fig:confusion} shows that \tractor\, is biased towards EXP sources. About 95\% of truly exponential sources are modeled as exponential, compared to only 20\% of truly \dev\, being modeled as \dev. This bias is surprisingly because \tractor\, model selection penalizes EXP and DEV sources equally \citep{obiwanMethods}.
%Which simulated ELGs does \legacypipe\, detect and model? Are there biases and systematics? How do the \tractor\, measured properties differ from input truth? 

\begin{figure}
 \includegraphics[width=\columnwidth]{confusion_matrix_by_type}
 \caption{Confusion matrix showing the fraction of truly exponential or \dev\, sources that \tractor\, models as type PSF, SIMP, EXP, DEV, or COMP sources. \tractor\, is biased towards EXP sources because 95\% of truly exponential sources are modeled as exponential, compared to only 20\% of truly \dev\, being modeled as \dev. The other 80\% of truly \dev\, sources are classified as SIMP (50\%), EXP (20\%), and PSF (10\%). This bias is surprisingly because \tractor\, model selection penalizes EXP and DEV sources equally \citep{obiwanMethods}.}
 \label{fig:confusion}
\end{figure}

Fig. \ref{fig:tractor-uncertainty} shows the number of standard deviations away from truth of \tractor's \gb, \rband, \zb\, flux, $\rhalf$, and ellipticity e1, e2 measurements. There is a large systematic offset between \tractor\, and truth for \gb, \rband, \zb\, flux and $\rhalf$. \tractor\, fluxes are too faint and $\rhalf$ too large. We believe that this is due to improper sky subtraction (see Section \ref{a:biases-systematics}). \tractor ivar underestimates the true error on all measured quantities: by 1.75-2x for \gb, \rband, \zb\, flux, 3-5x for $\rhalf$, and 2.7-3x for e1, e2. 

\begin{figure*}
\begin{center}
\begin{tabular}{ccc}
      Flux & $\rhalf$ & e1, e2 \\
      % list of figures goes left to right, then down a row, ...
      \includegraphics[width=0.3\textwidth]{num_std_dev_gaussfit_flux_separate_plots_g} &
      \includegraphics[width=0.3\textwidth]{num_std_dev_gaussfit_rhalf_bytype_EXP_submean} &
      \includegraphics[width=0.3\textwidth]{num_std_dev_gaussfit_e1_e2_EXP} 
      \\
      \includegraphics[width=0.3\textwidth]{num_std_dev_gaussfit_flux_separate_plots_r} &
      \includegraphics[width=0.3\textwidth]{num_std_dev_gaussfit_rhalf_bytype_DEV_submean} &
      \includegraphics[width=0.3\textwidth]{num_std_dev_gaussfit_e1_e2_DEV} 
      \\
      \includegraphics[width=0.3\textwidth]{num_std_dev_gaussfit_flux_separate_plots_z} &
      \includegraphics[width=0.3\textwidth]{num_std_dev_gaussfit_rhalf_bytype_SIMP_submean} &
      \includegraphics[width=0.3\textwidth]{num_std_dev_gaussfit_e1_e2_SIMP}
      \\
\end{tabular}
\end{center}
\caption{Number of standard deviations away from truth of \tractor's \gb, \rband, \zb\, flux, $\rhalf$, and ellipticity e1, e2 measurements. There is a large systematic offset between \tractor\, and truth for \gb, \rband, \zb\, flux and $\rhalf$. \tractor\, fluxes are too faint and $\rhalf$ too large. We remove the offset subtracting the mean and believe that this is due to improper sky subtraction (see Section \ref{a:biases-systematics}). (Left) \tractor's \gb, \rband, \zb\, flux errors are 1.75-2x larger than \tractor ivar implies and appear Gaussian distributed. (Middle) \tractor\, $\rhalf$ for sources classified as EXP and DEV. \tractor\, $\rhalf$ errors appear Gaussian distributed in both cases, but are 3x larger than \tractor's ivar implies for EXP sources and 5x larger for DEV sources. 
%SIMP, \tractor's ivar is now an overestimate by a factor of 4. Most sources have $\rhalf \sim 0.5$\arcsec, but not all (Fig. \ref{fig:rhalf-by-type}). 
(Right) \tractor ellipticity (e1, e2) errors for EXP and DEV sources. Unlike \gb, \rband, \zb\, flux and $\rhalf$, the systematic offset is very small and probably due to chance variation. The e1, e2 errors are identical and appear Gaussian distributed. \tractor\, ivar underestimates the true error by 3x and 2.7x for EXP and DEV sources. 
    \label{fig:tractor-uncertainty}}
\end{figure*}

Fig. \ref{fig:tractor-uncertainty} averages out any dependence on \gb, \rband, \zb\, magnitude, so we plot the number of standard deviations between truth and \tractor\, flux versus \gb, \rband, \zb\, magnitude in Fig. \ref{fig:num-std-dev-and-dmag}. We also show the magnitude residuals. The number of standard deviations of the systematic offset is constant with magnitude, while the magnitude difference is larger for fainter sources.

\begin{figure*}
\begin{center}
\begin{tabular}{ccc}
      Number of Standard Deviations & Magnitude Difference \\
      % list of figures goes left to right, then down a row, ...
      \includegraphics[width=0.4\textwidth]{num_std_dev_vs_grzmag_bytype_all} &
      \includegraphics[width=0.4\textwidth]{dmag_vs_grzmag_bytype_all}
      \\
\end{tabular}
\caption{(Left) Number of standard deviations away from truth of \tractor's \gb, \rband, \zb\, flux measurements (assuming \tractor\, ivar) versus \gb, \rband, \zb\, magnitude. Yellow lines are the q25, 50, 75 percentiles. The number of standard deviations of the systematic offset is constant with magnitude. We find identical results when remaking this figure for only PSF, SIMP, EXP, and DEV sources, respectively. (Right) Magnitude difference between \tractor's flux measurements and truth. Relative bright sources have a smaller magnitude difference (interquartile range of 0.1-0.2) than relatively faint sources (interquartile range of 0.4-0.5 mag).}
\label{fig:num-std-dev-and-dmag}
\end{center}
\end{figure*}

We say \legacypipe\, recovers a injected source when there is exactly one \tractor\, catalogue source within 1\arcsec of an injected source, and no DR3 \tractor\, catalogue sources within 1\arcsec of the injected source. We chose a 1\arcsec matching radius because \obiwan\, injects sources at the nearest pixel center. This should lead to an RA and Dec offset, between the true centroid and \tractor's measurement, equal to the DECaLS pixel scale of 0.262\arcsec / pixel. Fig. \ref{fig:delta-radec} shows that the offset is larger (0.4\arcsec) but still small enough that 1\arcsec is fine. We think the larger offset is due to co-registry and \tractor's simultaneous fitting of multiple images.

\begin{figure}
 \includegraphics[width=7 cm]{delta_dec_vs_delta_ra}
 \caption{RA and Dec residuals (arcsec) between the true centroid and \tractor's measurement of it. There is a systematic offset of 0.4\arcsec because \obiwan\, injects sources at the nearest pixel center. The offset would be equal to the DECaLS pixel scale, 0.262\arcsec / pixel, but co-registry and simultaneous fitting of multiple images makes it larger.}
 \label{fig:delta-radec}
\end{figure}


%\obiwan\, provides a \tractor-independent measurement for depth. It is the magnitude of the source for which the chance of recovery (i.e. detecting it then deciding that it is a bonafide astrophysical source) is 50\% . Fig. \ref{fig:fraction-recovered} shows the fraction of exponential $\rhalf=0.5$\arcsec galaxies that are recovered by \legacypipe\, versus source magnitude for subset 60. The fraction is never less than 0.5 so subset 60 is more than 0.5 mag deeper than full-depth in all bands. This is good for our images of the COSMOS region, but bad for DECaLS because any conclusions that the DECaLS Team has derived from subset 60 will be too optimistic. 

% We can also ask, what magnitude source in these images becomes a 5\sigma detection with \tractor? (this depends on tractor flux ivar so is not independent of tractor)


\section{Conclusions}
\label{sec:conclusions}

\red{ADD TEXT}
%\begin{itemize}
%\item this is the way to assess bias/systematics when doing joint analysis of datasets from multiple telescopes
%\end{itemize}


\section*{Acknowledgements} 
\label{sec:ack}

Funding for the DEEP2 Galaxy Redshift Survey has been provided by NSF grants AST-95-09298, AST-0071048, AST-0507428, and AST-0507483 as well as NASA LTSA grant NNG04GC89G.
 
 
%%%%%%%%%%%%%%%%%%%%%%%%%%%%%%%%%%%%%%%%%%%%%%%%%%

%%%%%%%%%%%%%%%%%%%% REFERENCES %%%%%%%%%%%%%%%%%%

% The best way to enter references is to use BibTeX:

\bibliographystyle{mnras}  %{mnras} if file is mnras.bst
\bibliography{bib} % {bib} if file is bib.bib


% Alternatively you could enter them by hand, like this:
% This method is tedious and prone to error if you have lots of references
%\begin{thebibliography}{99}
%\bibitem[\protect\citeauthoryear{Author}{2012}]{Author2012}
%Author A.~N., 2013, Journal of Improbable Astronomy, 1, 1
%\bibitem[\protect\citeauthoryear{Others}{2013}]{Others2013}
%Others S., 2012, Journal of Interesting Stuff, 17, 198
%\end{thebibliography}

%%%%%%%%%%%%%%%%%%%%%%%%%%%%%%%%%%%%%%%%%%%%%%%%%%

%%%%%%%%%%%%%%%%% APPENDICES %%%%%%%%%%%%%%%%%%%%%

\appendix

\section{Injecting Realistic ELGs}

\subsection{eBOSS ELG Targets}
\label{a:input-sample}

We construct our ``eBOSS'' galaxy sample from the joint tables of \tractor\, catalogues and eboss21, 22, 23 spectra as follows. We assume that all sources \tractor \, classifies as type PSF are compact or unresolved galaxies. We reclassify them as EXP with $\rhalf = \text{avg}(\text{FWHM}) / 2)$, where the average is over all bands. We also reclassify SIMP sources as EXP. We drop COMP sources because these comprise less than 1\% of the sample. 
%keep galaxies having reasonable \tractor\, measured $\rhalf$.

DEV galaxies are systematically larger and about 1 mag brighter than EXP in all bands (see Fig. \ref{fig:dev-brighter}), so we split the above sample into separate DEV and EXP samples. We also drop all NGC galaxies (about 30\% of the sample) because the NGC \tractor catalogues are not complete to \gb $< 23.8$ mag (see Fig. \ref{fig:ngc-incomplete}), 0.025 - 0.1 mag brighter than the \gb\, limit for eBOSS ELGs. This yields 77,525 EXP and 7,439 DEV galaxies. We refer to this as our eBOSS sample. The full list of cuts we apply is:
%9\% of the original sample is DEV, but they fill in parameter space enough for use to model. 
\begin{itemize}
\item !NGC
\item \verb|z_flag == 1| 
\item $0 \le \text{redshift} \le 2$
\item \verb|brick_primary|
\item $\text{type} \neq \text{COMP}$
\item $\rhalf  > 0.131$\arcsec\, (Nyquist limit, one half of the DECam pixel scale)
\item (EXP) $\rhalf  < 2.5$\arcsec
\item (DEV) $\rhalf  < 5.0$\arcsec
\end{itemize}

\begin{figure}
 \includegraphics[width=\columnwidth]{dev_brighter}
 \caption{PDFs of \gb, \rband, \zb mag (top) and redshift and $\rhalf$ (bottom) for eBOSS ELG targets that are spectroscopically confirmed to be galaxies. DEV galaxies (red) are systematically larger and about 1 mag brighter than EXP galaxies, in all bands. We reclassify PSF sources as EXP with $\rhalf = \text{avg}(\text{FWHM}) / 2)$, where the average is over all bands.}
 \label{fig:dev-brighter}
\end{figure}

\begin{figure}
 \includegraphics[width=5 cm]{ngc_incomplete}
 \caption{The eBOSS NGC footprint is not deep enough to detect eBOSS ELGs with $g < 22.825$ mag. We remove galaxies in the NGC region from our sample of representative ELGs.}
 \label{fig:ngc-incomplete}
\end{figure}

Because the SGC \gb \, mag PDF in Fig \ref{fig:ngc-incomplete} is a step function for $g > 22.825$, a Gaussian Mixture model for the brightness, shape, and redshift distribution of eBOSS ELGs will be inaccurate at the faint \gb \, mag, where most of the ELGs reside. To model the full distribution of ELG properties we bootstrap sample from our eBOSS sample.

\subsection{eBOSS ELG Almost-Targets}
\label{a:eboss-almost-targets}

We construct our DR3-DEEP2 sample as follows. We match DEEP2 galaxies to the nearest DR3 tractor catalogue sources using a 1\arcsec\, matching radius and keeping the nearest neighbor. We reclassify SIMP and PSF sources as EXP the same way we did for the eBOSS sample (see Section \ref{a:input-sample}), and then apply the following cuts, 
\begin{itemize}
\item !NGC
\item INSERT DEEP2 FLAG THAT SAYS THE OBJECT IS A GALAXY
\item $0 \le \text{redshift} \le 2$
\item \verb|brick_primary|
\item \gb, \rband, \zb $\text{flux} > 0$
\item \gb, \rband, \zb $\text{flux ivar} > 0$
\item $\text{type} \neq \text{COMP}$
\item $\rhalf  > 0.131$\arcsec\, (Nyquist limit, one half of the DECam pixel scale)
\item (EXP) $\rhalf  < 2.5$\arcsec
\item (DEV) $\rhalf  < 5.0$\arcsec
\end{itemize}

We find that the DR3-DEEP2 sample reproduces the brightness, shape, and redshift distributions of the eBOSS sample. Fig. \ref{fig:dr3dp2-reproduces} shows the PDFs of \gb, \rband, \zb, redshift, and $\rhalf$ for EXP galaxies, and that the DR3-DEEP2 galaxies (cut to eBOSS ELG targets) agree very well with the eBOSS sample. Fig. \ref{fig:dr3dp2-redshift} shows that for EXP galaxies, \gb, \rband, \zb, and $\rhalf$ depend on redshift in the same way for the DR3-DEEP2 galaxies (cut to eBOSS ELG targets) and the eBOSS sample. A similar level of agreement occurs for DEV galaxies.

\begin{figure}
 \includegraphics[width=\columnwidth]{dr3dp2_reproduces}
 \caption{PDFs of \gb, \rband, \zb, redshift, and $\rhalf$ for EXP galaxies. The DR3-DEEP2 galaxies (blue), cut to type EXP galaxies, the SGC footprint, and eBOSS ELG targets, reproduces the brightness, shape, and redshift distributions of the eBOSS sample (red).}
 \label{fig:dr3dp2-reproduces}
\end{figure}

\begin{figure}
 \includegraphics[width=\columnwidth]{dr3dp2_redshift_trends}
 \caption{For EXP galaxies, \gb, \rband, \zb, and $\rhalf$ depend on redshift in the same way for the DR3-DEEP2 galaxies (red) and the eBOSS sample (blue). SWAP COLORS SO BLUE IS DR3-DP2 LIKE PREV FIG}
 \label{fig:dr3dp2-redshift}
\end{figure}

Because DR3-DEEP2 galaxies are representative of eBOSS ELGs and extend beyond the eBOSS selection boundaries, we can construct a sample of ELG-like galaxies to test contamination into the eBOSS selected sample. We choose to keep DR3-DEEP2 SGC galaxies within a padding of 0.2 mag of the eBOSSS selection boundaries, yielding a sample of 1,064 EXP and 85 DEV galaxies. \cite{obiwanMethods} show that \tractor\, flux measurements are accurate to 0.1 - 0.2 mag for \gb $\sim 22.9$ galaxies, so a padding of 0.2 mag should capture most contaminants due to \tractor\, measurement errors. Fig. \ref{fig:dr3dp2-vs-eboss} compares our eBOSS and DR3-DEEP2 samples for EXP and DEV galaxies.

\begin{figure}
     \subfloat[type EXP\label{fig:dr3dp2-vs-eboss-exp}]{%
       \includegraphics[width=\columnwidth]{dr3dp2_vs_eboss_exp}
     }
     \hfill
     \subfloat[type DEV\label{fig:dr3dp2-vs-eboss-dev}]{%
       \includegraphics[width=\columnwidth]{dr3dp2_vs_eboss_dev}
     }
\caption{PDFs of \gb, \rband, \zb mag, $\rhalf$, and redshift for galaxies from the eBOSS (blue) and DR3-DEEP2 (red) samples. (Top) EXP galaxies. (Bottom) DEV galaxies. The DR3-DEEP2 sample includes galaxies within 0.2 mag of the eBOSS ELG target selection boundaries, so it is brighter and fainter than the eBOSS sample. The DR3-DEEP2 DEV are noisy because the sample size is so small (85 galaxies).}
\label{fig:dr3dp2-vs-eboss}
\end{figure}


\subsection{eBOSS ELGs and n(z)}
\label{a:final-eboss-sample}

We now describe our algorithm to jointly sample redshift from the eBOSS n(z) and the associated \gb, \rband, \zb flux and shape from our eBOSS and DR3-DEEP2 samples. The eBOSS n(z) is the redshift distribution from the \verb|eBOSS.ELG.obiwan.eboss*.fits| tables weighted by spectroscopic completeness (1/TSR). The NGC imaging data are incomplete, so we drop NGC spectra. We sample from the resulting n(z) by intentionally over-fitting a 10 component Gaussian Mixture Model (GMM) to it (see Fig. \ref{fig:nz}).

\begin{figure}
     \subfloat[eBOSS SGC\label{fig:nz-eboss}]{%
       \includegraphics[width=0.49\columnwidth]{nz_eboss}
     }
     \hfill
     \subfloat[10 component GMM\label{fig:nz-eboss-fit}]{%
       \includegraphics[width=0.46\columnwidth]{nz_eboss_and_thefit}
     }
\caption{(Left) n(z) for eBOSS SGC based on eBOSS 21, 22, and 23 spectra. (Right) Overplotting the PDF from 10,000 draws from our 10 component GMM for n(z), which intentionally over-fits the data.}
\label{fig:nz}
\end{figure}

We draw redshifts from n(z), dropping those outside the allowed redshift range [0,2], until there are N redshift samples. Each sample gets a unique id, which is an integer [1,N] that we call ``id". For each redshift, we find the nearest redshift in our EXP-DEV combined DR3-DEEP2 sample. This is a n(z)-weighted draw from ELG-like galaxies within 0.2 mag of the eBOSS ELG SGC target selection boundaries. We define an ELG as passing eBOSS ELG SGC target selection. If the redshift-brightness-shape sample is an ELG, we find its nearest redshift in the EXP eBOSS sample (90\% of the time) or the DEV eBOSS sample (10\% of the time); if not, we trim the DR3-DEEP2 sample to galaxies that extends beyond the eBOSS ELG selection boundaries, and find its nearest redshift in the trimmed EXP DR3-DEEP2 sample (90\% of the time) or the trimmed DEV DR3-DEEP2 sample (10\% of the time). This yields a sample of eBOSS ELG-like galaxies with the desired redshift distributed. Fig. \ref{fig:final-eboss-sample-10k} shows the PDFs of \gb, \rband, \zb, redshift, and $\rhalf$ for 10,000 draws using the above algorithm.

\begin{figure}
 \includegraphics[width=\columnwidth]{final_eboss_elg_sample_10k_draws}
 \caption{PDFs of \gb, \rband, \zb, redshift, and $\rhalf$ for 10,000 draws using the algorithm in Section \ref{sec:final-eboss-sample}. \obiwan\, will inject galaxies having these properties into the set of DECam CCDs used to produce DR3-era \tractor\, catalogues and the eBOSS ELG target lists.}
 \label{fig:final-eboss-sample-10k}
\end{figure}

We add a random RA and Dec coordinate, positions angle, minor to major axis ratio to each of the N draws. We assign an id saying where the brightness and shape information came from, which we call \verb|id_sample|. This is the SDSS-ID (\verb|plate-mjd-fiberid|) if from the EXP or DEV eBOSS samples or the \tractor-ID (\verb|brickid-objid|) if from the  EXP or DEV DR3-DEEP2 samples. 

\section{Biases and Systematics in \legacypipe}
\label{a:biases-systematics}

This section identifies the bad sources and other biases and systematics that we remove from our \obiwan\, eBOSS data set to arrive at the figures presented in Section \ref{sec:biases-systematics}. Fig. \ref{fig:dflux-systematic} shows the number of standard deviations between true flux and \tractor\, measurement before bad sources at $x=0$ are removed or the mean is subtracted. The bad sources are cleanly removed with {\tt{fracin}} $< 0.2$ in the \tractor catalogues and are reported as ``not real'' by the Legacy Survey\footnote{\url{http://legacysurvey.org/dr6/files}}. 

\begin{figure}
     \subfloat[Distribution including bad sources and systematic offset\label{fig:fracin-present}]{
       \includegraphics[width=0.45\columnwidth]{num_std_dev_gaussfit_flux_bytype_all_notsubmean}
     }
     \hfill
     \subfloat[Bad sources are linearly separable: {\tt{fracin}} $< 0.2$\label{fig:fracin-2dhist}]{
       \includegraphics[width=0.45\columnwidth]{fracin_vs_numstddev_2dhist}
     }
     \hfill
     \subfloat[Bad sources only\label{fig:fracin-dflux}]{
       \includegraphics[width=0.45\columnwidth]{num_std_dev_gaussfit_flux_bytype_all_keepwhatputin_fracin_keep_bad_020_notsubmean}
     }
     \hfill
     \subfloat[Distribution after removing bad sources and subtracting the mean\label{fig:fracin-removed}]{
       \includegraphics[width=0.45\columnwidth]{num_std_dev_gaussfit_flux_bytype_all_keepwhatputin_fracin_020}
     }
\caption{The input distribution of \tractor\,, truth residuals (top left) and the bad sources and systematic offsets we remove to arrive at the true distribution of \tractor\,, truth residuals (bottom right). (Top Left) Number of standard deviations including bad sources at x = 0, and systematic offset to x < 0. (Top Right) The excess of sources with flux residual of zero are bad sources. We find that these have {\tt{fracin}} $< 0.2$ in the \tractor\, catalogues and are linearly separable from all other sources. (Bottom Left) Number of standard deviations for bad sources ({\tt{fracin}} $< 0.2$) (Bottom Right) Number of standard deviations after removing bad sources and subtracting the mean.}
\label{fig:dflux-systematic}
\end{figure}

Fig. \ref{fig:safe-to-remove} shows that it is safe to remove the sources with {\tt{fracin}} $< 0.2$. Their \gb, \rband, \zb\, magnitude, $\rhalf$, and redshift distributions are nearly identical to that of {\tt{fracin}} $\ge 0.2$ sources. And this cut only removes about 11\% of the sample, independent of whether the source is truly exponential or \dev.

\begin{figure}
 \includegraphics[width=0.8\columnwidth]{hist_all_quantities_fracin_cut}
 \caption{(Bottom right) Fraction of injected exponential and \dev\, galaxies that are recovered by \legacypipe\, (orange) or remain after \legacypipe\, recovery and the cut on fracin $< 0.2$. The fracin cut removes 11\% of the recovered exponential and 11\% of the recovered \dev\, galaxies. These are similar fractions so the fracin cut should not introduce any systematics.}
 \label{fig:safe-to-remove}
\end{figure}

Fig. \ref{fig:confusion} shows that \legacypipe\, is biased towards EXP sources. We find that this is not a consequence of source size. Fig. \ref{fig:rhalf-recovered} shows the distribution of $\rhalf$ for sources classified as PSF, SIMP, EXP, and DEV. Fig. \ref{fig:confusion} shows that \legacypipe\, is biased towards EXP sources, but since true $\rhalf$ distributions for type EXP and DEV are so similar, it is unlikely that the bias is due to source size. In fact, Fig. \ref{fig:rhalf-recovered} shows that for $\rhalf > 1.5$\arcsec the chance of recovery by \legacypipe\, drops to and fluctuates about 50\%. And this is independent of whether classifies the source as EXP or DEV.

\begin{figure}
 \includegraphics[width=0.7\columnwidth]{hist_true_rhalf_by_type}
 \caption{PDFs of true $\rhalf$ for sources that \legacypipe\, classifies as PSF, SIMP (top) and EXP, DEV (bottom). The most common injected source size is $\rhalf \sim 0.5$\arcsec (Fig. \ref{fig:input}), and this is also the most common recovered source size independent of \tractor\, type (not just SIMP). PSF and SIMP sources are generally smaller than EXP and DEV. However, type DEV sources have a higher fraction of small ($rhalf < 0.5$\arcsec) sources than SIMP or EXP. The true $\rhalf$ distributions for type DEV and EXP are otherwise very similar, which means that the larger \tractor\, errors on $\rhalf$ for type DEV (see Fig. \ref{fig:tractor-uncertainty}) are not due to source size. SIMP sources have the highest fraction of $\rhalf \sim 0.5$\arcsec sources sizes, which probably explains why \tractor\, errors on $\rhalf$ and e1, e2 can be so small. }
 \label{fig:rhalf-input}
\end{figure}

\begin{figure}
 \includegraphics[width=0.7\columnwidth]{fraction_recovered_vs_rhalf}
 \caption{How the fraction of sources recovered by \legacypipe\, depends on the size of the source. $\rhalf \sim 1.5$\arcsec is the critical size after which the chance of recovery drops to and fluctuates about 50\%.}
 \label{fig:rhalf-recovered}
\end{figure}

%\begin{figure}
% \includegraphics[width=\columnwidth]{grz_hist_by_type}
% \caption{PDFs of true \gb, \rband, \zb magnitude for type EXP, DEV, PSF, and SIMP sources (blue, green, cyan, and magenta, respectively). (Left) EXP and DEV sources with SIMP for comparison. EXP and DEV have nearly identical PDFs, in all bands. The PDFs for EXP and DEV drop to zero for faint sources, and this drop coincides with an increase in the SIMP PDF for faint sources. (Right) PSF and SIMP with EXP for comparison. In \rband\, and \zb, the EXP, SIMP, and PDFs have similar shape but PSF is located about 0.25 mag fainter than SIMP and SIMP is located about 0.25 mag fainter than EXP.}
% \label{fig:grz-hist-type}
%\end{figure}

%\section{The \obiwan DR5 Run}
%
%\subsection{Details of the DR5 Run}
%
%we ran \obiwan\, on about 500 deg$^2$ of the DR5\footnote{\url{http://legacysurvey.org/status/}} footprint
%
%
%\subsection{Injecting Realistic ELGs for DESI}
%
%We match DR3 to DEEP2 cutting to only spectroscopically confirmed galaxies. 
%  
%Emission Line Galaxies (ELGs) are blue star forming galaxies that are spectroscopically identified by a strong OII emission line. The DEEP2 Galaxy Redshift Survey (DEEP2, references) collected one of the largest samples of ELG spectra to date, since the spectral resolution resolves the OII emission line. To model the real population of ELGs, we match our DR3 Tractor catalogues to the DEEP2 catalogues using a 1\arcsec matching radius.
%
%This gives us a sample of ELGs with high confidence redshift (from DEEP2 spectra), brightness (g, r, z flux), and shape (half light radii, ellipticity) information. While the sample is biased by the DEEP2 selection function, the depth of DR5 imaging, and \tractor's ability to measure the centroid, brightness, and shape, this is as good as we can do with publicly available data. The best possible sample is probably to use Hyper Suprime-Cam Subaru Survey (HSC-SS) imaging in the DEEP2 fields (Hayashi et al. 2017, Mehta et al. 2017).
%
%We model the redshift-brightness-shape parameter space with Gaussian Mixture Models (GMMs).
%
%\dev comprise less than 3\% of the DESI sample, and do not model them because the sample size is too small to capture the underlying distributions in redshift and brightness. For DESI, we apply the following cuts,
%
%{\it{ALL}}
%\begin{itemize}
%\item flux $>$ 0, ivar $>$ 0
%\item rename ``EXP, REX, and SIMP'' as EXP
%\item rename ``PSF" as EXP but with rhalf= avg(psfsize g,r,z) / 2
%\item rhalf $<$ 5
%\end{itemize}
%
%{\it{DESI + EXP (DEV, COMP dropped)}}
%\begin{itemize}
%\item redshift >= 0.8-0.2 and redshift <= 1.4 + 0.2, so we don�t fit a gaussian to a step function
%\item 0.5 mag padding perpendicular to the TS box
%\item 0.5 mag deeper than than g,r,z depth limits
%\item fit GMM to [fwhm-OR-rhalf, g-r, r-z,z,redshift], 8 components doesn't overfit but captures correlations between variables
%\item draw N points, drop points with LogicalOr (0.8 < redshift > 1.4, more than 0.5 deeper than g, r, z limits, and rhalf > pixscale/2), redraw as many points as dropped, repeat until have N points
%\end{itemize}
%
%We model the DESI n(z) for ELGs by over-fitting a GMM model to the desired n(z) for DESI. We randomly sampled 100,000 points from the Sanchez \& Kirkby ELG mocks, then fit a 12 component GMM to the resulting n(z) since this reproduced the n(z) very well.
%
%The procedure for generating the sample of ELGs with representative properties for DESI is as follows. We have GMM models for n(z) and the joint distribution of redshift, brightness, and shape. Draw N samples from the n(z) model. Draw 10k samples from the redshift, brigthness, shape model, filtering out points outside the redshift or detection limits. Organize the 10k samples in a KD Tree. For each of the N samples, find the nearest redshift in the 10k sample, and assign the corresponding brightness and shape. This becomes the sample we inject into the imaging data. It has the exact n(z) for DESI or eBOSS and reasonable brightnesses and shapes for each redshift. 42\% of the DESI sample is in the DESI TS box. Note, we repeated the above procedure using DR2 and DR3, respectively, instead of DR5 and ended up with similar Gaussian Mixture Models.

\section{Data Products}
\label{sec:data-products}

TRANSFER DATA TO eBOSS PROJECT DISK SPACE SAY HOW PEOPLE CAN ACCESS IT


%%%%%%%%%%%%%%%%%%%%%%%%%%%%%%%%%%%%%%%%%%%%%%%%%%


% Don't change these lines
\bsp	% typesetting comment
\label{lastpage}
\end{document}

% End of mnras_template.tex
